\documentclass[a4paper, 10pt]{article}
\title{Repurposing a Sampling-Based Planner for a Six-Degree-of-Freedom Manipulator to Avoid Synthetic Moving Obstacles}

\usepackage{authblk}

%small text for affilliation
\renewcommand\Affilfont{\itshape\small}

\author[1]{Hafiz Iman}
\author[1]{Raisuddin Khan}

\affil[1]{Department of Mechatronics Engineering, Kulliyah of Engineering, International Islamic University Malaysia}
  
%formattingi
  \newcommand\thisPaperDir{/home/asl/version-control/ws_thesis/writing_papers/resampling_planning_in_dynamic_environment}
  \newcommand\thesisDir{/home/asl/version-control/ws_thesis/thesis_latex}
  \usepackage{import}
  %\usepackage{natbib}
  \usepackage[backend=biber,bibencoding=utf8,style=authoryear]{biblatex}
  \addbibresource{\thesisDir/bibtex/library.bib}
	\usepackage{soul,color}
  \usepackage{pdflscape} %for changing page orientation
  \usepackage[shortlabels]{enumitem}
  \usepackage{amssymb}
  \usepackage{graphicx} %for inserting photo
  \usepackage{subcaption}
  \usepackage{amsmath} % for equations
  \DeclareMathOperator*{\argmax}{argmax} % for underword argmax

  \usepackage{mathtools} % for equations
  \usepackage{physics}
  \usepackage{arydshln} %adding dash line to matrix

  \usepackage{float}  %deactivate this if hyperref does not work with algo
  \usepackage{hyperref}
  \usepackage{tocbibind}  
  \usepackage[linesnumbered,ruled,vlined]{algorithm2e}
  \usepackage{titlesec} %for deeper subsections
  \usepackage{multirow} %for merging row in cell
  \setcounter{secnumdepth}{4} %setting subsection 4 section deep
  % numbering equations
  \numberwithin{equation}{section} %equation number based on chapter

  \usepackage{xfrac} % for fraction in the for n/d
  \usepackage{esint} % for easy integral symbols

  %for listing algorithms from algorithm2e
  \newcommand{\listofalgorithmes}{\tocfile{\listalgorithmcfname}{loa}}
  %for footnote
  \renewcommand{\thefootnote}{\fnsymbol{footnote}}

  % for formattgin algorithm2e box [iman] I have to put after listofalgorithmes
  \newcommand\mycommfont[1]{\footnotesize\ttfamily\textcolor{blue}{#1}}
  \SetCommentSty{mycommfont}
  \SetKwInput{KwInput}{Input}                % Set the Input
  \SetKwInput{KwOutput}{Output}              % set the Output

  % This command is useful when importing subdirectory items

%  %margin
\usepackage{setspace}
\usepackage[
  a4paper,
  left=3.8cm,
  right=2.5cm,
  top=2.5cm,
  bottom=3cm,
  footskip=1.7cm
]
{geometry}

%IIUM font size
%syntax {\TITLEfontsize IIUM idiotize thesis formatting }
\newcommand\TITLEfontsize{\fontsize{17pt}{20.4pt}\selectfont}
\newcommand\CHAPTERfontsize{\fontsize{14pt}{16pt}\selectfont}
%For line over line for signature
\usepackage{calc}
\newcommand{\sigline}[1]{\makebox[\widthof{#1~}]{.\dotfill}\\#1}
%another over line signature command
%syantax \sign{The signant name} 
%syntax \Date but use it within minipage
%\begin{minipage}[t]{0.4\linewidth}
%    \raggedright
%    \sign{Supervisor}
%    \par
%    Mr.\,L. L. Silva\par
%    Department of Computing and Information Systems, \par
%    Faculty of Applied Sciences, \par
%    University of Moratuwa
%  \end{minipage}%
% \hfill
%  \begin{minipage}[t]{0.4\linewidth}
%    \sign{Signature of the supervisor}
%    \Date
%  \end{minipage}
\newcommand{\sign}[1]{%      
  \begin{tabular}[t]{@{}c@{}}
  \makebox[1.5in]{\dotfill}\\
  \strut#1\strut
  \end{tabular}%
}
\newcommand{\Date}{%
  \begin{tabular}[t]{@{}c@{}}
    \makebox[0.85in]{\dotfill}\\
    \strut Date \strut
  \end{tabular}%
}

%paragraph
\parindent=12mm


%header style
\usepackage{fancyhdr}
%\cfoot{\thepage}
\pagestyle{plain}
%\fancyhead{}
%\fancyhead[C]{\nouppercase{\textit{\leftmark}}} %puts chapter title on even page in lower-case italics
%\fancyhead[CO]{\nouppercase{\textit{\rightmark}}} %puts section title on odd page in lower-case italics
%\renewcommand{\headrulewidth}{0pt} %gets rid of line
\renewcommand{\chaptermark}[1]{\markboth{#1}{}} %gets rid of chapter number
\renewcommand{\sectionmark}[1]{\markright{#1}} %gets rid of section number

%making sure table and figures caption is 1 single spacecaption is 
%make sure that the tables are not in \center environment
%it will add more space between caption and the table/figures
\usepackage{caption}
\captionsetup[table]{skip=12pt}

%adding frames to a page for copygright
\usepackage{mdframed}
\usepackage{verbatim}

%using font close to Times New Roman
\usepackage{mathptmx}
% since mathptmx changes \mathcal symbols 
% use this patch to retained the default one
\DeclareMathAlphabet{\mathcal}{OMS}{cmsy}{m}{n}


\usepackage{titletoc}

%provide control over the appearance of table of contents
%figures and others
\usepackage[titles]{tocloft}
%adding dotted or dot leaders to chapter in table of content (toc)
%\renewcommand{\cftchapfont}{\bfseries}
%\renewcommand{\cftchappagefont}{\bfseries}
\renewcommand{\cftchapleader}{\cftdotfill{\cftdotsep}} % for chapters
% prepend CHAPTER on chapter number in toc
\renewcommand{\cftchappresnum}{CHAPTER }
% add ':' separator between chapter number and chapter title
\renewcommand{\cftchapaftersnum}{:}
\setlength{\cftchapnumwidth}{6.5em}
% making section normal font instead of bold
%adding "Table" keyword to the list
%provide abit of space so that the 'Table X.XX'
%keyword does not overlap with the table caption
\renewcommand{\cfttabpresnum}{Table\ }
\setlength{\cfttabnumwidth}{5.5em}

%adding the "Figure" keyword to the list
%provide abit of space so that the 'Figure X.XX'
%keyword does not overlap with the figure caption
\renewcommand{\cftfigpresnum}{Figure\ }
\setlength{\cftfignumwidth}{5.5em}

% Annotation is being displayed in bibliography/references
% This will suppress it
\DeclareSourcemap{
  \maps[datatype=bibtex]{
    \map{
      \step[fieldset=annote,     null]
      \step[fieldset=annotation, null]
    }
  }
}


%margin
\usepackage{setspace}
\usepackage[
  a4paper,
  left=3.8cm,
  right=2.5cm,
  top=2.5cm,
  bottom=3cm,
  footskip=1.7cm
]
{geometry}

%chapter section style
\usepackage{titlesec} %for deeper subsections and changing chpt sec style
\titleformat{\chapter}
   [display] %possible values are hang, leftmargin, drop, etc.
   {\fontsize{14}{16}\bfseries\centering} % format to be applied to the whole title
   {\MakeUppercase{\chaptertitlename} \thechapter} %label setting
   {24pt} %horizontal separation between label and title body
   {\MakeUppercase} %code preceding the title body
   \titlespacing{\chapter}{0pt}{0pt}{48pt} %4spacing after title


\setcounter{secnumdepth}{4} %setting subsection 4 section deep
%paragraph
\parindent=12mm


%header style
\usepackage{fancyhdr}
\pagestyle{fancy}
\fancyhead{}

%making sure table and figures caption is 1 single spacecaption is 
%make sure that the tables are not in \center environment
%it will add more space between caption and the table/figures
\usepackage{caption}
\captionsetup[table]{skip=12pt}
  % This command is for keywords
  \providecommand{\keywords}[1]{\textbf{\textit{Keywords---}} #1}

  \date{} %setting date as empty string so no date shown in the article

  \newcommand\rimini[0]{\textit{r\_mini}} %r_mini typesetting

  %subfigure referecing

\begin{document}
{\let\newpage\relax\maketitle}
\begin{abstract}
  This paper presents the use of a sampling-based planner 
  as a local planning scheme
  to avoid obstacle between a robotic arm and a moving obstacle.
  Based on a planner benchmark on a obstacle-ridden environment,
  rapidly-exploring random tree (RRT) planner are used to 
  populate the trajectories of the
  task space and map them into a configuration space using
  Newton-Raphson-based inverse kinematic solver.
  Three robot poses are defined in a cycle of back-and-forth motion; 
  the starting, the midpoint, 
  and the end pose. The starting pose and the end pose are equal.
  The robot repeatedly moves from the starting pose to the end pose via the 
  midpoint pose. Poses between the three are a subset of a trajectory populated 
  by the sampling-based planner. Each set of trajectory is unique. 
  We impose
  periodically occuring synthetic obstacle that moves in and out of 
  the robot arm workspace defined in a simulated environment. 
  Within the robot's workspace, 
  the obstacle moves and cuts through the cyclical space to emulate a 
  dynamic environment. Based on the performance of the local
  planning strategy, the robot has a higher success in
  avoiding the obstacles when the planning query starts during the presence of 
  the obstacles in the cyclical space. However, a higher chance of collision
  occurs when the planning query is invoked when the obstacle nearly 
  approaches the cyclical space.
\end{abstract}

\keywords{mechatronics, robotics, planner, path-planning, sampling-based planner, robotic arm, manipulator, dynamic environment, moving obstacles, ROS, MoveIt}
  


\section{Introduction}

Robot manipulators such as industrial robots 
work well in repetitive and heavy tasks. 
They are high-performing, objective, and relentless at
task that are too difficult to complete by an operator or
a group of workers. However, given their rigid and massive construction, even
a small-sized industrial robot impose significant hazards on the people that work near it. 
Hence, recently, robot manipulators are more
compliant and are designed to work with workers cooperatively without risking 
their safety \parencite{Hagele2016}. 
Regardless, it is still an issue of hazard should a compliant or cooperative
robot collide with a person working at close to 
its workspace \parencite{Matthias2011}. 
The collision also warrants expensive maintenance and repairs. To address a more
intelligent motion and, evidently, a safer motion planning, 
an industrial robot system implements a certain degree
of planning algorithm specifically for the motion of the manipulator.

A robot motion planner provides a
collision-free motion solution for a manipulator.
Here, a solution is defined as collections of waypoints or trajectories. 
In the case of the traditional definition of industrial robots, 
the planning is global because the robot is enclosed and isolated
in workcell. A global planner takes in a set of initial and goal 
positions, $t \in \mathbb{R}^3$, or as set of initial and goal poses, $p \in \vb*{SE(3)}$, as its
input and generates constraint-informed trajectory as intermediate waypoints
for the robot to follow. 
However, a global planner is offline, which implies that the trajectories 
are set before the task commences. The global planner also 
assumes a static workspace. 
Any unplanned changes in the workspace over the global planning-scheme, 
such as an unplanned introduction of a  stationary object or 
a moving object into the robot workspace, renders the 
offline-planned trajectory outdated and, consequently, requires replanning. 
In the case
of a compliant robot, unplanned changes are unavoidable.

Hence, it
is imparative for a compliant manipulator or cooperative manipulator 
to have an efficient motion planner because replanning is computationally expensive and time-consuming; 
sampling-based planners are the pinnacle example
of efficient planning \parencite{Elbanhawi2014}. 
Unfortunately, sampling-based planner
trade completeness and optimality with efficiency where %cite Kravaki1998 
the planner may fail to
provide a solution \parencite{Kavraki1998}. Also, if a solution exists 
under its metric space, 
the waypoints may 
not be the least cost path to a goal \parencite{LaValle1998}. 
Regardless of lack of
completeness and optimality, sampling-based planner
excel at maintaining reasonable usage of computational
resources that pave way to the near-online planning scheme.

The sampling-based planners for robot motion
are a family of planners that
uses probabilistic approach to generate a graph structure
encoding the free space and the robot configuration space.
The sampling are stochastic, such that resampling
will give a unique solution to the previous sampling. Most 
sampling-based planner are also tractable in higher-dimensional
configuration and task space. However, 
sampling-based planners assume static workspace. 
%cite kravaki1996 and introducing probabilistic planner
%cite LaValle and introduce randomly exploring 

%cite all sampling-based planners in OMPL here as an example.

%[check if this is correct] 
%However, known usage of sampling-based planners are in the context of global
%planning. 

%write about reactive planner cite all the planners 
%discussed in Kulich2015: there are three reactive planners
 

%write about the lack of planning platform with easy plugins
%write about the general approach to planning in dynamic environment pipeline figure

This paper repurposes the use of sampling-based motion planning, the
rapidly-exploring random tree motion planning, 
to address operation task in a dynamic environment in 
Euclidean space, $\mathbb{R}^3$. Our method 
leverage the efficiency and the computationally reserved 
sampling-based motion planning without needing to apply purely reactive
motion planning approach so that computational resources can 
be delegated to other tasks, i.e., motion-tracking, state estimation, 
mapping, localization, and motion control. 
In the following of this report, we will assume sampling 
planners provide
solution in higher dimensional configuration space, which implicates
a solution with a set of poses represented by the special Euclidean group,
$\vb*{SE(3)}$.

This paper is a part of a collision avoidance system for a compliant
robot manipulator design to prove an encoderless concept, and only
cover the formulation of the planning approach we adopt for our 
robotic system. 
%cite figure here. 
%attached figure of the pipeline of this research 

\section{Related Work}

In this review, we will look into research papers 
that delve 
into motion planning in a dynamic environment for robot manipulators in three
dimensional space, $\mathbb{R}^3$. We pay close attention to 
algorithms derived from sampling-based planner which use stochastic
approach to query the configuration space. 
The planning algorithms for robot motion in a dynamics environment
dated back to 1985, albeit mainly applied on
mobile robot \parencite{Mohanan2018}. 

Some planning algorithms that is non-probabilistic 
for robotic arm 
are also observed which are
based on 
representation space \parencite{Su2011,Liu2016a}, 
sequential expanded Langrangian homotopy \parencite{Dharmawan2018a}, and
real-time adaptive motion planning (RAMP) \parencite{Vannoy2008}. These 
works provide a new method of planning and tackle
the problem at the local planning and at lower level of control to 
solve problems with the aid of a simulated environment. 
Their simulated
environment, or a map model of the environment, includes dynamic objects
that are then filtered and disregarded in the map registration pipeline.
Their experimentation approach will be adopted in this 
research where, the map is informed
with the presence of obstacles that are moving in the workspace without
having another sub-module of the system tracking the object via 
motion tracking.


%cite Kunz2016
%cite
\subsection{The Probabilistic Motion Planning}
\textcite{Kavraki1996} is the first group of researchers 
that used probability model for sampling 
the configuration space for holonomic robot motion 
such as a manipulator robot. The planner are called 
the probabilistic roadmap (PRM) motion planning. The algorithm
construct a graph structure 
to find path between an initial pose to a goal pose in
two-dimensional configuration space, $n=2$. \textcite{Kavraki1996} also
proof a more general solution for higher dimensional configuration space,
$n>2$. With graph structure,
more than one path connect the initial pose to the goal pose. Therefore, PRM is a 
multi-query type planner. 

\textcite{Kunz2010a} improve PRM by redefining the distance metric 
of a robot manipulator so that the robot can move
around a moving obstacle in real-time.
Their approach performs well in an uncluttered environment. 
\textcite{Kunz2010a} also redefined 
the distance function of the PRM to address dynamic objects, such as
a walking person, into a two-dimensional map. Although the 
configuration space of the manipulator is in $\mathbb{R}^3$, the map
, constructed from a two-dimensional LiDAR scan, is in $\mathbb{R}^2$.

%RRT introduction here
In retrospect, the RRT was formulated for 
non-holonomic motion \parencite{LaValle1998} targeting
problems addressed in diffential-constrained motion such
as a car on a plane. However, given the model of its metric space
and consequently the configuration space, RRT are tractable
for higher dimensional problem such as manipulator motion in 
3D space \parencite{Wei2018}. RRT assume as static
environment but \textcite{Wei2018} successfully 
change the way RRT samples a robot metric space so that
it is fast enough to react with a changing environment. Also, unlike PRM,
RRT works well in a cluttered environment because of 
the randomized sampling on the robot configuration space in 
the metric space. 

Researchers have been modifying the PRM 
\parencite{Klasing2007,Likhachev2005,Jaillet2004, Pomarlan2013a} 
and the RRT
\parencite{Otte2015,Ferguson2007,Ferguson2006,Bekris2007} to facilitate
better performance. 
Unlike \textcite{Kunz2010a} and \textcite{Wei2018}, so few have applied their planning
algorihtms on a robot manipulator despite both algorithms
provides mathematical framework for .


We will use the method demonstrated by \textcite{Kunz2010a} and
\textcite{Wei2018}
to design our experiment of a moving obstacle collision avoidance with
the implementation of the vanilla RRT to solve motion for robot
manipulator in three-dimensional space, $\mathbb{R}^3$. Different
from the implementation by \textcite{Wei2018} our method
implement the vanilla
RRT where we do not represent the obstacle configuration space.

\section{Formulation and Algorithms}

This paper
will use the superscript notation to refer the control space and the 
subscript as the equivalent representation in the configuration space.
For example, $C^{ee}$, refers to the control space of the end-effector
where the controlling pipelines would take in
$c^{ee} = (\theta_{1},\ldots,\theta_{6}) \in C^{control}$, 
and the equivalent pose is in the configuration space, $C_{ee}$. Since, 
revolute joint topology is the 1-hypersphere, $S^1$, we will
assume that, for the case of 6R robot,
it's joints are limited to a certain range which makes $c^{ee} \in \mathbb{R}^{6\times1}$.

\subsection{The Geometry of a Compliant Robotic Arm, \rimini}

We prototype and build a 3D-printed robot called 
Richard Mini (\rimini, see \textit{figure \ref{fig:r_mini_hardware}}) 
based
on the conditioned addressed by \textcite{Pieper1968}, which 
entails three collated joints sharing the same cross 
point of their $z-axes$ shown in \textit{figure \ref{fig:r_mini_pieper}}. 

\import{\thisPaperDir/figures/}{r_mini_hardware.tex}

\import{\thisPaperDir/figures/}{r_mini_pieper.tex}

\rimini~has six revolute axes,
$(\theta_1,\ldots, \theta_6)$. The
first three axes moves the task space from one point to another
representing translation vector, $t \in \mathbb{R}^3$. 
The last three axes of 
the manipulator rotate the task space representing the
rotation operation about the task space frame, $R \in \mathbb{R}^{3\cross3}$. 
Hence 
the complete transformation of the task space via
the joint movement is represented by the homogenous 
transformation matrix, $T \in \vb*{SE(3)}$, where 
$\vb*{SE(3)}$ is homomorphic to $(R,\vb*{t})$; 
$\vb*{SE(3)} = \mathbb{R}^3  \cross \vb*{SO(3)}$.
The matrix representation of the homogenous transformation 
is shown in \textit{equation \ref{mat:transformation_matrix}}

\import{\thisPaperDir/equations/}{transformation_matrix.tex}

The kinematic models of the \rimini follows the Denavit-Hartenberg (DH) 
formulation \parencite{Denavit1955}. The DH-parameters are shown 
in \textit{table \ref{tab:rimini_dh_table}} and the visualization of these
parameters in the form of frames transformation are shown in 
\textit{figure \ref{fig:rimini_joints}}.

\import{\thisPaperDir/equations/}{rimini_dh_table.tex}
\import{\thisPaperDir/figures/}{rimini_joints.tex}

Following the DH-convention, each links rotates about the $z-axis$ of
each frames it's attached to where $(joint_1, \ldots, joint_6)$
correspond to $(\theta_1, \ldots, \theta_6)$ respectively.
In
\textit{table \ref{tab:rimini_dh_table}}, each row of 
represents a homogenous transformation(\textit{equation \ref{eq:homogenous_transformation}})

\begin{equation}\label{eq:homogenous_transformation}
  T_i = T(R_{z}(\theta_i))T(t_{z}(d_i)) T(t_{x}(a_i)) T(R_{x}(\alpha_i))
\end{equation}

where, $R_z,R_x \in \vb*{SO(3)}$, are the rotation operations 
about the $z-axis$ and the $x-axis$ respectively, and 
$t_z, t_x \in \mathbb{R}^3$ are the translation vector
on the $z-axis$ and the $x-axis$ respectively, 
while, $d,a,\alpha \in \mathbb{R}$, are scalars. 
The homogenours transformation between the origin
of the base of the robot into the end-effector of the 
robot, which coincide with $joint_6$ is shown in
\textit{equation \ref{eq:base_ee_transformation}},

\import{\thisPaperDir/equations/}{base_ee_transformation.tex}

where
$t_{ee} \in \mathbb{R}^3$, is the point location of the 
end effector in 3D space,
and $t_0 \in \mathbb{R}^3$ is the origin of the 
base of the robot. Since the rotation involve 
in \textit{equation \ref{eq:base_ee_transformation}} involves
roation about the origin of the local frames,
the orientation of the end-effector can also be represented by,

\import{\thisPaperDir/equations/}{ee_rotation.tex}

where the operation is closed. Now, the rotation operation
of \rimini \    can be represented by quaternion,

\import{\thisPaperDir/equations/}{ee_q_rotation.tex}

where an
\textit{Equations \ref{eq:base_ee_transformation}} and \textit{{\ref{eq:ee_rotation}}}
represent the forward kinematic solution for the end-effector of \rimini. 
 

The self collision, robot collision checking is delegated to 
a collision and proximity query library based on the
implementation by \textcite{Pan2012}. Later in the algorithm formulation 
of the RRT and the cycle space, subroutine from
the flexible collision library (FCL) library will be invoke to check collisions between
the manipulator and the moving obstacles encoded
inside the collision scene inside the planning scene ($\mathcal{M}$) when the RRT datastructure
is initialized (refer to algorihtm \ref{algo:RRT} line \ref{ln:initialize_tree}).

We use Newton-Raphson method to find the inverse kinematic solution of 
\rimini, $\hat{c}^{ee}$. The generalization of the method uses the current value of the 
robot's encoder, $c^{current}$, and the termination value, $\epsilon=0.005$,
to end the iteration. Algorithm \ref{algo:newton_raphson} delineate the 
the method.

\import{\thisPaperDir/algorithms/}{newton_raphson_table.tex}

%how mapping from task space to configuration space
%how jacobian is populated
\subsection{The Rapidly-Exploring Random Trees and its Mathematical Background} 
This research uses RRT implementation provided by OMPL library
packaged in the MoveIt software. The algorithm for the purpose 
of this research is shown in \textit{algorithm \ref{algo:RRT}}:

\import{\thisPaperDir/algorithms/}{RRT_table.tex}

where $k$ represent the, number of node in the 
tree generated by the RRT, $\mathcal{M}$, represent the collision map
which is part of the planning scene where all
RRT sampling takes place and $\mathcal{T}$ is the tree 
that points to a non-colliding space. In this RRT implementation, 
the map are loaded or queried in line \ref{ln:initialize_tree} 
each time the \textit{generateRRT()} is invoked.
Line~\ref{ln:random_generator} generates a random state bias towards
the $C_{goal}$.
Line~\ref{ln:knn} invokes the k-nearest neighbor to find a selection 
of nodes that is closes to the state configuration, $c_{random}$.
Line~\ref{ln:direction} is the important part of the sampling 
in the RRT where it represent the controlling input of the
robot motion. Since, the robot are control in the joint-configuration
space, the angular joint limit address the shape
of the workspace. However, given the angular velocity, 
these limits are translated into 
the configuration space via the kinematic Jacobian which requires
the information on the $\Delta t$. The limits implicitly ensures 
that the RRT, by executing line~\ref{ln:direction} within the context 
of the robots Jacobian, does not pass through the singularities of 
the robot. Hence, the configuration space of the manipulator 
also includes, $C_{limit}$, containing configuration
that abides the joint-configuration space range and angular velocity
limit.

The configuration space where 
the RRT-sampling is concerned is modified in this paper where, 
the rotation representation and its sampling is in
$R \in \mathbb{H}$, such that the parameterization of
the Hamiltonian-space is the quaternions,
$\vb*{q} \in \mathbb{R}^4$. Therefore, the representation of the robot poses
and also the non-colliding poses, $(\vb*{q},\vb*{t})$ 
are explained in \textit{equation \ref{eq:configuration_vector}}.

\import{\thisPaperDir/equations/}{configuration_vector.tex}

The RRT sampling involves query into a map, that stores collision objects.
This is the planninq scene, denoted as collision map, where the RRT sampling occurs.
The query are
called when both initial pose and a goal pose are sent to the RRT planner input. 
The output of the pipeline is a set of non-colliding space where from there another
pipeline, transform the configuration space into a control space. We will define
the control space in the following section.
  % matrix t/q of the C_pose = (t,q)
%write the algorithm here
%read section 2 of Lavelle1998
\subsection{The Cyclical Space}
The cyclical space is the subset of the planner solution where the RRT algorithm is invoke
twice. 
During the generation of the cyclical space,  the RRT
output the a trajectory from the
initial pose, $c_{initial}$ to the goal pose, $c_{goal}$, 
into a controlling pipeline. The trajectory
are then map from the configuration space into the joint-configuration space via
the Newton-Raphson inverse kinematic solver (see algorithm \ref{algo:newton_raphson}. To complete 
the set of the cyclical space, the entries in the initial pose and the goal pose
are swapped, while invoking query into the collision map,
\import{\thisPaperDir/equations/}{trajectory.tex}

which forms a cyclical motion between the initial pose and the goal
pose. Here, 
$C_{cycle} = \overbrace{ C_{initial \rightarrow goal}}^\text{see algorithm \ref{algo:cycle_space} line \ref{ln:cs_RRT1}} 
              + \overbrace{C_{goal \rightarrow initial}}^\text{see algorithm \ref{algo:cycle_space} line \ref{ln:cs_RRT2}}$.
Algorithm~\ref{algo:cycle_space} block explains how the $C_{cycle}$ space is constructed.

\import{\thisPaperDir/algorithms/}{cycle_space_table.tex}



The control space are represented by the trajectory in the joint-configuration
space $C^{control} \subset C^{cycle}\in S $
where $S$ is the 6-hypersphere. In \textit{equation \ref{eq:trajectory}} the
joint-configuration space are equivalent with the configuration space in 
\textit{equation \ref{eq:configuration_vector}}. 
The control space
is the direct controlling parameters for the movement of 
\rimini where it
only handles the control space (or joint-state space).
The sampling of the RRT to generate the tree data structure, $\mathcal{T}$,
are done within the $\vb{SO(3)} \times \mathbb{R}^3$ topology. The free
configuration space, or the non-colliding pose, are represented by, 
$C_{free} = C_{workspace} / C_{obstacle}$. According to 
\textcite{LaValle1998}, the $C_{obstacle}$ are also covers 
the physical contraint of the non-holonomic movement of the robot. 
However, in the case of an articulated robot arm in this research
, the configuration 
limitation are the range of the joints and the angular velocity limit. 
Since, all of these
measurements are in the n-hyperspace, to map them into the
$\vb*{SO(3)} \times \mathbb{R}^3$, we use the kinematic Jacobian.
% THe implicit singularity, which discuss jacobian matrix, velocity limit
% and acceleration limit and , det(J) = 0
% Our approahc select the limit of velocity based on 
% the capacity of the motor. And choosing a reasonable acc 


%Why PRM is not the best candidate for this
%method: Because it requires re-learning a scene
% when a robot reach the midpoint. Re-learning
% requires replanning and the construction of roadmap
% hence computationally more taxing because re-learnig
% implies previous roadmap are not usable and requires
% fresh construction of roadmap. The concern we have
% with this repeating roadmap construction is
% on the nature of the algorihm which requires
% normalizing the 
%However, for RRT, the re-learning and resampling is
% not as heavy as PRM because, unlike PRM, the 
% datastructure of the tree is faster during initial 
%scene learning.

%write about local planner and the global planner
%write the algorithm here

\section{Methodology}
\subsection{Benchmarking of Sampling-Based Motion Planners}

In this research, the planner for the dynamic obstacle
avoidance are selected based on the performance of a benchmarking
activity. Here, the procedure is explained.

Two poses are set for the benchmark, pose initial are represented in the 
form of \textit{equation \ref{eq:configuration_vector}}. 
The following vectors explain the numerical
value of these poses with respect to the frame attached to the base
of \rimini.

\import{\thisPaperDir/equations/}{benchmark_pose_vector.tex}

A box, with dimension, 0.5 m $\times$ 0.05 m $\times$ 0.575 m, are place
in front of the robot, it's pose is described by the vector,

\import{\thisPaperDir/equations/}{box_pose.tex}

{\textit{Figure \ref{fig:benchmark_simulation}} shows the simulation 
setup and the planning motion in action. 
\import{\thisPaperDir/figures/}{benchmark_simulation.tex}
The simulation is ran for 50 request from the initial pose to 
the goal pose. Time processing is given a 10 s limit. The memory limit
is set to 1 Mb. The time limit for a request, including the motion and the
processing time is set to 3637 s. This paper use these configurations and
the default configuration of each planners in the MoveIt to start the benchmarking. 
%cite kunz here and explain 200ms that is used as the benchmark
% for human reaction time against the time for a planner to complete
% it's searching. 
\subsection{Experiment Design}
The cyclical space is populated by the RRT-Newton-Raphson
pipeline where the generated trajectories are
then pass to the control pipeline where
the controller will spline the sparse trajectory
waypoints. Two poses are defined in this experimentation which
has been described in \textit{equation \ref{eq:benchmark_pose_vector}}.
A moving obstacle are placed in front-view of the robot. The obstacle
is a cylinder with 0.1 m radius base at 1 m height. 
The obstacle moves from 0.3 m to 1.7 m away from the robot
in oscillation. The period of motion is harmonic, such that, 
the robot follows a $1+0.7\sin(2\pi(0.0006)\Delta t)$. Two velocities 
values were used: 50\% and 10\% scale from the maximum velocity 
of the end-effector. 

The planner is invoked 5 s before the obstacle is placed 
into the planning scene. The cylinder
are directly place into the planning scene such that no motion 
tracking is necessary for this research. The planner are requested to
provide solution for the motion described by the cycle space. 
Five iterations are done with each given a five minutes runtime. The metric
use for this experiment is the time on first collision where,
when the prototype touches the cylinder, the iteration is terminated. 
This experimentatation
is done, both, in simulation, and with
the real robot hardware. However, for both the simulation and the hardware
validation, the obstacle are augmented in simulated environment.
% the scope and the assumption of the experiment design
% discuss what is the performance level of the planner
% constrain on each joints
% constraint on joint velocity
% constraint on joint acceleration
% show simulation setup without moving obstacle
% show simulation setup with moving obstacle
% the dimension of the moving obstacle and what it represent
\section{Result and Discussion}
There are two part of the result on this paper, the first dealing
with the benchmarking result to ascertain the best 
sampling-based planner. The second part delve
into the performance of the selected planner 
from the benchmarking on a moving obstacle. 
\subsection{Benchmark Result}
\import{\thisPaperDir/figures/}{benchmarking_process_time.tex}
\textit{Figure \ref{fig:benchmarking_process_time}} shows the 
compiled statistics of the time the solution were pass to the 
controller (in this case a fake controller for simulation of 
\rimini \ in the simulated environment). RRT requires on average,
0.031 planning time while PRM requires 0.035 planning time from
the initial pose to the goal pose when subjected to an obstacle
very close to the robot. 
\textcite{Wei2018} explained the improved RRT algorithms, such as 
the bi-RRT, and the RRT-connect, 
solve a query faster. However, based on our benchmarking, 
vanilla RRT, or base-RRT, and PRM outperform their improved variants when
completing the path query between an initial pose and a goal pose. 
To that end, this research uses vanilla RRT as the scheme for the 
high-level local planner. 
This result helps in selecting the
motion planner for the dynamic obstacle avoidance. 

\subsection{The Performance of RRT on a Dynamic Environment}

\import{\thisPaperDir/equations/}{experiment_table.tex}

Table \ref{tab:experiment_table}, shows the recorder time to
collision of 20 iterations. The average time to collision is 40 s. 
There are two iterations where there are no collision recorded. This 
poor performance is subjected to the algorithm \ref{algo:cycle_space},
specifically in line \ref{ln:cs_RRT1} and line \ref{ln:cs_RRT2}, RRT is called.
Within this call (algorithm \ref{algo:RRT}, line \ref{ln:initialize_tree} consider
an obstacle map that is outdated 
given the cyclinder has moving further towards the manipulator when RRT: 
line~\ref{ln:initialize_tree}. 
Within the RRT algorithm, there are no mechanism for the robot to stop or move at 
a lower rate to avoid the cylinder. \textit{Figure \ref{fig:obs_avoidance_fail}} shows
the sequence when the end-effector collide with the cylinder. 

\import{\thisPaperDir/figures/}{obs_avoidance_fail.tex}

Despite the obstacle avoidance fails when the 
moving cylinder approaches the robot specifically 
when the centroid of the cylinder is nearing
the $x-axes$ of the $c_{initial}$ and $c_{goal}$,
the planner successfully avoid the obstacles when 
the lines
\ref{ln:cs_RRT1} and \ref{ln:cs_RRT2} in
algorithm \ref{algo:cycle_space} is invoked. 

\import{\thisPaperDir/figures/}{avoiding_moving_obs.tex}

\import{\thisPaperDir/figures/}{obs_avoidance.tex}

The planner shows reactive behavior when 
the cyclical space is initialize, via algorithm \ref{algo:cycle_space}.
.\textit{Figure \ref{fig:avoiding_moving_obs}} illustrates
such behavior in the simulated environment, and \textit{figure \ref{fig:obs_avoidance}}
shows the same behavior in the hardware reiteration of the experimentation. This is 
illustrated in \textit{ figure \ref{fig:reaction_joint}}, where the $joint_2$ and $joint_3$
changes the range of their movement while $joint_3$ changes the rate of its movement. 

\import{\thisPaperDir/figures/}{reaction_joint.tex}

No significant changes are observed for $joint_4, joint_5$ and $joint_6$. 
This is the implication of 
the Pieper-condition manipulator design where, none of the $z-axis$ from the first three
joints shares the same crossing point, which suggest the rotation acting by these joints 
are not a linear transformation. Due to the offset (affine transformation) of the joints' 
axis of rotation, these joints' there is a bijection mapping of these joints
to the task-space. Also changes are also observed on the orientation
of the frame attached to the end-effector, however, there are no bijection mapping of
the three joints to the task-space's orientation.

\section{Conclusion and Recommendation}
This concludes that, RRT on a dynamic setup
is capable at reacting to an obstacle when the obstacle 
is moving,however, the RRT on a dynamic obstacle imposed under
the cyclical space is not satisfactory.
Given this reactive behavior for a repurposing
a sampling based planner,  it is recommended that this algorithm should be improved 
by including more than one intermediate poses between the initial pose  and the goal pose. 
Also, an implementation on the tracking of the obstacle, by means of
obstacle state estimation, would have helped 
the performance of the planner.

\clearpage
\printbibliography

\end{document}


