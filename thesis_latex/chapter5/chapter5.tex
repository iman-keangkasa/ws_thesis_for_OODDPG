\providecommand{\setflag}{\newif \ifwhole \wholefalse}
\setflag %create new if variable name ifwhole(whole)  
\ifwhole\else %if whole flag is set to true do nothing else do codes below

  \documentclass[12pt,a4paper,oneside,hidelinks]{book}

    %\input{extra_package.tex}
    %\input{tweak.tex}
    %\input{commando.tex}
    %\input{font}
  \newcommand\thesisDir{/home/asl/version-control/ws_thesis/thesis_latex}
  \usepackage{import}
  %\usepackage{natbib}
  \usepackage[backend=biber,bibencoding=utf8,style=authoryear]{biblatex}
  \addbibresource{\thesisDir/bibtex/library.bib}
	\usepackage{soul,color}
  \usepackage{pdflscape} %for changing page orientation
  \usepackage[shortlabels]{enumitem}
  \usepackage{amssymb}
  \usepackage{graphicx} %for inserting photo
  \usepackage{subcaption}
  \usepackage{amsmath} % for equations
  \DeclareMathOperator*{\argmax}{argmax} % for underword argmax

  \usepackage{mathtools} % for equations
  \usepackage{physics}


  \usepackage{float}  %deactivate this if hyperref does not work with algo
  \usepackage{hyperref}
  \usepackage{tocbibind}  
  \usepackage[linesnumbered,ruled,vlined]{algorithm2e}
  \usepackage{titlesec} %for deeper subsections and changing chpt sec style
  \setcounter{secnumdepth}{4} %setting subsection 4 section deep
  % numbering equations
  \ifwhole
    \numberwithin{equation}{chapter} %equation number based on chapter
  \else
%    \numberwithin{equation}{section} % equation number based on section
  \fi

  \usepackage{xfrac} % for fraction in the for n/d
  \usepackage{esint} % for easy integral symbols

  %for listing algorithms from algorithm2e
  \newcommand{\listofalgorithmes}{\tocfile{\listalgorithmcfname}{loa}}
  %for footnote
  \renewcommand{\thefootnote}{\fnsymbol{footnote}}

  % for formattgin algorithm2e box [iman] I have to put after listofalgorithmes
  \newcommand\mycommfont[1]{\footnotesize\ttfamily\textcolor{blue}{#1}}
  \SetCommentSty{mycommfont}
  \SetKwInput{KwInput}{Input}                % Set the Input
  \SetKwInput{KwOutput}{Output}              % set the Output

  % This command is usefule when importing subdirectory items
  %margin
\usepackage{setspace}
\usepackage[
  a4paper,
  left=3.8cm,
  right=2.5cm,
  top=2.5cm,
  bottom=3cm,
  footskip=1.7cm
]
{geometry}

%IIUM font size
%syntax {\TITLEfontsize IIUM idiotize thesis formatting }
\newcommand\TITLEfontsize{\fontsize{17pt}{20.4pt}\selectfont}
\newcommand\CHAPTERfontsize{\fontsize{14pt}{16pt}\selectfont}
%For line over line for signature
\usepackage{calc}
\newcommand{\sigline}[1]{\makebox[\widthof{#1~}]{.\dotfill}\\#1}
%another over line signature command
%syantax \sign{The signant name} 
%syntax \Date but use it within minipage
%\begin{minipage}[t]{0.4\linewidth}
%    \raggedright
%    \sign{Supervisor}
%    \par
%    Mr.\,L. L. Silva\par
%    Department of Computing and Information Systems, \par
%    Faculty of Applied Sciences, \par
%    University of Moratuwa
%  \end{minipage}%
% \hfill
%  \begin{minipage}[t]{0.4\linewidth}
%    \sign{Signature of the supervisor}
%    \Date
%  \end{minipage}
\newcommand{\sign}[1]{%      
  \begin{tabular}[t]{@{}c@{}}
  \makebox[1.5in]{\dotfill}\\
  \strut#1\strut
  \end{tabular}%
}
\newcommand{\Date}{%
  \begin{tabular}[t]{@{}c@{}}
    \makebox[0.85in]{\dotfill}\\
    \strut Date \strut
  \end{tabular}%
}

%paragraph
\parindent=12mm


%header style
\usepackage{fancyhdr}
%\cfoot{\thepage}
\pagestyle{plain}
%\fancyhead{}
%\fancyhead[C]{\nouppercase{\textit{\leftmark}}} %puts chapter title on even page in lower-case italics
%\fancyhead[CO]{\nouppercase{\textit{\rightmark}}} %puts section title on odd page in lower-case italics
%\renewcommand{\headrulewidth}{0pt} %gets rid of line
\renewcommand{\chaptermark}[1]{\markboth{#1}{}} %gets rid of chapter number
\renewcommand{\sectionmark}[1]{\markright{#1}} %gets rid of section number

%making sure table and figures caption is 1 single spacecaption is 
%make sure that the tables are not in \center environment
%it will add more space between caption and the table/figures
\usepackage{caption}
\captionsetup[table]{skip=12pt}

%adding frames to a page for copygright
\usepackage{mdframed}
\usepackage{verbatim}

%using font close to Times New Roman
\usepackage{mathptmx}
% since mathptmx changes \mathcal symbols 
% use this patch to retained the default one
\DeclareMathAlphabet{\mathcal}{OMS}{cmsy}{m}{n}


\usepackage{titletoc}

%provide control over the appearance of table of contents
%figures and others
\usepackage[titles]{tocloft}
%adding dotted or dot leaders to chapter in table of content (toc)
%\renewcommand{\cftchapfont}{\bfseries}
%\renewcommand{\cftchappagefont}{\bfseries}
\renewcommand{\cftchapleader}{\cftdotfill{\cftdotsep}} % for chapters
% prepend CHAPTER on chapter number in toc
\renewcommand{\cftchappresnum}{CHAPTER }
% add ':' separator between chapter number and chapter title
\renewcommand{\cftchapaftersnum}{:}
\setlength{\cftchapnumwidth}{6.5em}
% making section normal font instead of bold
%adding "Table" keyword to the list
%provide abit of space so that the 'Table X.XX'
%keyword does not overlap with the table caption
\renewcommand{\cfttabpresnum}{Table\ }
\setlength{\cfttabnumwidth}{5.5em}

%adding the "Figure" keyword to the list
%provide abit of space so that the 'Figure X.XX'
%keyword does not overlap with the figure caption
\renewcommand{\cftfigpresnum}{Figure\ }
\setlength{\cftfignumwidth}{5.5em}

% Annotation is being displayed in bibliography/references
% This will suppress it
\DeclareSourcemap{
  \maps[datatype=bibtex]{
    \map{
      \step[fieldset=annote,     null]
      \step[fieldset=annotation, null]
    }
  }
}


  
  %This is for group notation
   \newcommand\thisPaperDir{/home/iman/version-control/ws_thesis/writing_papers/resampling_planning_in_dynamic_environment}
 
\begin{document}

\fi


\chapter{Experimentation, Result and Discussion}\label{chap:experimentation_result_discussion}

\section{Benchmarking Experiment Design and Result on the Sampling-Based Planner in Static Environment}

In this research, the planner for the dynamic obstacle
avoidance are selected based on the performance of a benchmarking
activity. Here, the procedure is explained.

Two poses are set for the benchmark represented in the 
form of \textit{equation \ref{eq:configuration_vector}}. 
The following vectors explain the numerical
value of these poses with respect to the frame attached to the base
of \rimini.

\import{\thisPaperDir/equations/}{benchmark_pose_vector.tex}

A box, with dimension, 0.5 m $\times$ 0.05 m $\times$ 0.575 m, are place
in front of the robot, it's pose is described by the vector,

\import{\thisPaperDir/equations/}{box_pose.tex}

{\textit{Figure \ref{fig:benchmark_simulation}} shows the simulation 
setup and the planning motion in action. 
\import{\thisPaperDir/figures/}{benchmark_simulation.tex}
The simulation ran for 50 request from the initial pose to 
the goal pose. Time processing is given a 10 s limit. The memory limit
is set to 1 Mb. The time limit for a request, including the motion and the
processing time is set to 3637 s. This thesis uses these configurations and
the default configuration of each planners in the MoveIt to start the benchmarking. 
%cite kunz here and explain 200ms that is used as the benchmark
% for human reaction time against the time for a planner to complete
% it's searching. 

\import{\thisPaperDir/figures/}{benchmarking_process_time.tex}
\textit{Figure \ref{fig:benchmarking_process_time}} shows the 
compiled statistics of the time the solution were passed to the 
controller (in this case a virtual controller for simulation of 
\rimini \ in the simulated environment). RRT requires on average,
0.031 planning time while PRM requires 0.035 planning time from
the initial pose to the goal pose when subjected to an obstacle
very close to the robot. 
\textcite{Wei2018} explained the improved RRT algorithms, such as 
the bi-RRT, and the RRT-connect, 
solve a query faster. However, based on our benchmarking, 
vanilla RRT, or base-RRT, and PRM outperform their improved variants when
completing the path query between an initial pose and a goal pose. 
To that end, this research uses vanilla RRT as the scheme for the 
high-level local planner. 
This result helps in selecting the
motion planner for the dynamic obstacle avoidance. 


\section{Experiment Design for Unpredictable Obstacles using RRT}

The cyclical space is populated by the RRT-Newton-Raphson and the
pipeline where the generated trajectories are
then pass to the control pipeline where
the controller will spline the sparse trajectory
waypoints. Two poses are defined in this experimentation which are described by 
the vectors in \textit{equation \ref{eq:rrt_prm_static_obs_pose_vector}}
and the \textit{figure \ref{fig:initial_goal_pose_static_obstacle}}


\import{\thesisDir/chapter5/equations/}{rrt_prm_static_obs_pose_vector.tex}
\import{\thesisDir/chapter5/figures/}{initial_goal_pose_static_obstacle.tex}

In this expermiment, the robot completes a set of ten cyclical space. 
A static object is introduced between a straight line that connects
the initial and the goal pose. 
The obstacle is introduced in the fourth cycle space. It is then removed at
the end of the seventh cycle. The obstacle can take any shape, size, and
position in
the workspace. However,
for this experimentation, the obstacle for the simulation is a cyclinder with
height 0.25 m and radius 0.1 m. The simulation physique engine frontend is
the Gazebo software with the ODE library as the engine's backend. In 
\textit{figure \ref{fig:gazebo_sim_rrt_obs_avoidance}}, the cylinder is
shown via the simulated Kinect sensor feedback, and the representation
of the cyclinder in the planning scene are represented by the green
voxel clustering. 

\import{\thesisDir/chapter5/figures/}{gazebo_sim_rrt_obs_avoidance.tex}

A book with height 0.23 m, length 0.18 m, and 
thickness 0.02 m was used for the hardware validation.

The simulation was performed and replicated with hardware validation. Five
iteration were repeated on both the simulation experimentation and
the hardware experimentation.

\section{Experiment Design for Moving Planning in Dynamic Environment}
The cycliccal space is populated only by the RRT-Newton-Raphson pipline.
Two poses are defined in this experimentation which
has been described in \textit{equation \ref{eq:benchmark_pose_vector}}.
A moving obstacle are placed in front-view of the robot. The obstacle
is a cylinder with 0.1 m radius base at 1 m height. 
The obstacle moves from 0.3 m to 1.7 m away from the robot
in oscillation. The period of motion is harmonic, such that, 
the robot follows a $x=1+0.7\sin(2\pi(0.0006)\Delta t)$. Two velocities 
values were used: 50\% scale and 10\% scale of end-effector's maximum velocity.

The planner is invoked 5 s before the obstacle is placed 
into the planning scene. The cylinder
is directly place into the planning scene such that no motion 
tracking is necessary for this research. The planner are requested to
provide solution for the motion described by the cycle space. 
Five iterations are done with each given a five minutes runtime. The metric
use for this experiment is the time on first collision where,
when the prototype touches the cylinder, the iteration is terminated. 
This experimentatation
is done, both, in simulation, and with
the real robot hardware. However, for both the simulation and the hardware
validation, the obstacle are augmented in simulated environment.

\section{Results on Planning for Static and Dynamic Obstacle}

There are three hardware validations performed. One demonstrate planning
when introduced with unpredictable static obstacle. Another regarding the 
planning under dynamic environment where a synthetic obstacle, moving
periodically, cuts through the line of planning
in the cyclical space, $C_{cycle}$. The third experimentation
involve the use of RTAB-Map impregnated with the \acrfull{PHASER} implementation
where the state estimation pipeline was 
swapped with the spherical harmonics approach.

\subsection{Result on Path Planning for Unpredictable Static Environment}

The RRT planners avoid the static obstacle reactively under the algorithm 
\ref{algo:cycle_space}.
Since replanning occurs at the initial and goal pose, collision between
the static obstacle and the robot manipulator can only plan to avoid obstacle
when the obstacles are placed before a half-cycle space is completed. The obstacle
are placed in between the straight line that connects the two poses so that
a reactive and successful collision checking, and collision avoidance 
capability
can be demonstrated. When the cycle space generator invoked, 
\rimini~ avoids the obstacle successfully as depicted in 
\textit{figure \ref{fig:joints_profiles_with_snapshots_prm}}.

\import{\thesisDir/chapter5/figures/}{joints_profiles_with_snapshots_prm.tex}

\subsection{Result on Path Planning Under Dynamic Environment}

\import{\thisPaperDir/equations/}{experiment_table.tex}

Table \ref{tab:experiment_table}, shows the recorded time-to-collision 
of 20 iterations. The average time to collision is 40 s. 
There are two iterations with no collision recorded. This 
poor performance is subjected to the algorithm \ref{algo:cycle_space},
specifically in line \ref{ln:cs_RRT1} and line \ref{ln:cs_RRT2}, RRT is called.
Within this call (algorithm \ref{algo:RRT}, line \ref{ln:initialize_tree} consider
an obstacle map that is outdated 
given the cyclinder has moving further towards the manipulator when RRT: 
line~\ref{ln:initialize_tree}. 
Within the RRT algorithm, there are no mechanism for the robot to stop or move at 
a lower rate to avoid the cylinder. \textit{Figure \ref{fig:obs_avoidance_fail}} shows
the sequence when the end-effector collide with the cylinder. 

\import{\thisPaperDir/figures/}{obs_avoidance_fail.tex}

Despite the obstacle avoidance fails when the 
moving cylinder approaches the robot specifically 
when the centroid of the cylinder is nearing
the $x-axes$ of the $c_{initial}$ and $c_{goal}$,
the planner successfully avoid the obstacles when 
the lines
\ref{ln:cs_RRT1} and \ref{ln:cs_RRT2} in
algorithm \ref{algo:cycle_space} is invoked. 

\import{\thisPaperDir/figures/}{avoiding_moving_obs.tex}

\import{\thisPaperDir/figures/}{obs_avoidance.tex}

The planner shows reactive behavior when 
the cyclical space is initialize, via algorithm \ref{algo:cycle_space}.
.\textit{Figure \ref{fig:avoiding_moving_obs}} illustrates
such behavior in the simulated environment, and \textit{figure \ref{fig:obs_avoidance}}
shows the same behavior in the hardware reiteration of the experimentation. This is 
illustrated in \textit{ figure \ref{fig:reaction_joint}}, where the $joint_2$ and $joint_3$
changes the range of their movement while $joint_3$ changes the rate of its movement. 

\import{\thesisDir/chapter5/figures/}{reaction_joint.tex}

No significant changes are observed for $joint_4, joint_5$ and $joint_6$. 
This is the implication of 
the Pieper-condition manipulator design where, none of the $z-axis$ from the first three
joints shares the same crossing point, which suggest the rotation acting by these joints 
are not a linear transformation. Due to the offset (affine transformation) of the joints' 
axis of rotation, these joints' there is a bijection mapping of these joints
to the task-space. Also changes are also observed on the orientation
of the frame attached to the end-effector, however, there are no bijection mapping of
the three joints to the task-space's orientation.

\subsection{Result on Obstacle-less Planning with SLAM}
The result in \textit{figure \ref{fig:slam_odometry}} explains the
output of the state odometry pipeline.  The $\mathbb{R}^3$ part of the 
task space, $C_{ee}$, is normalized to the origin $(0,0,0)$.

\import{\thesisDir/chapter5/figures/}{slam_odometry.tex}

As the robot arm moves, changing the pose of the task space,
velocity change in the movement together with the
angular velocity from the rotation change in the end-effector
collapses \acrshort{PHASER}'s state estimation despite the attempt to use the quaternion and the 
Hilbert's space to discern the correspondence of two 3D point clouds 
snapshots. In the figure, there are singularities on most of the process. However,
estimations are successfully broadcast between
4.3 s to 6.3 s, 16.1 s to 19.4 s, 32.8 s to 35.2 s, 44.2 s to 47.3 s, 56.2 
to 64.5 s.


The absence of odometry estimation output, where the pipeline is silence, populates 73.4\% of
the process. The state estimation for the PHASER implementation is intermittent.

The pose estimation from the SLAM pipeline and also the odometry, where the visual odometry are taken
into consideration during data fusion, are only appear between 57.2 s and  61.3 s as 
shown in \textit{figure \ref{fig:slam_odom_xy}}.

\import{\thesisDir/chapter5/figures/}{slam_odom_xy.tex}

In \textit{figure \ref{fig:slam_visualization}}, the outputs of the visual
odometry and the SLAM estimation are
presented by the light blue lines in the left pane of the screenshot. The right shows, 
the initial and goal pose of the hardware. 

\import{\thesisDir/chapter5/figures/}{slam_visualization.tex}

The SLAM application on the path planning for this research would not be able to give
a stream of pseudo-continuous data output from its pipeline. Since, an intermittent
nature of these result were shown from the data, it is observed that, when the 
state estimation is map into the joint-space configuration, and then feeded into 
the controller's feedback pipeline, the motor could not abide to the 
trajectory solution populated by the planner. Hence, currently, with 
PHASER implementation of the state estimation, the SLAM solution and the planning 
algorithm will not reconcile and will not work without having an intermediate 
pipeline that can fill up the gap between the one estimation to the next from the 
RTAB-Map. 

\section{Summary}
In this chapter, the result of four experimentations are presented. These results explain 
the capability of
\rimini \ working in a static environment (via the benchmarking) and in a dynamic 
environment where this research consequently test on the prototype. The later result shows
the feasibility of SLAM at estimating the task space of \rimini.
\ifwhole\else
	\printbibliography
	\end{document}
\fi


