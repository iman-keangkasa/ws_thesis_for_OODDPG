\providecommand{\setflag}{\newif \ifwhole \wholefalse}
\setflag %create new if variable name ifwhole(whole)  
\ifwhole\else %if whole flag is set to true do nothing else do codes below

  \documentclass[12pt,a4paper,oneside,hidelinks]{book}

    %\input{extra_package.tex}
    %\input{tweak.tex}
    %\input{commando.tex}
    %\input{font}
  \newcommand\thesisDir{/home/asl/version-control/ws_thesis/thesis_latex}
  \usepackage{import}
  %\usepackage{natbib}
  \usepackage[backend=biber,bibencoding=utf8,style=authoryear]{biblatex}
  \addbibresource{\thesisDir/bibtex/library.bib}
	\usepackage{soul,color}
  \usepackage{pdflscape} %for changing page orientation
  \usepackage[shortlabels]{enumitem}
  \usepackage{amssymb}
  \usepackage{graphicx} %for inserting photo
  \usepackage{subcaption}
  \usepackage{amsmath} % for equations
  \DeclareMathOperator*{\argmax}{argmax} % for underword argmax

  \usepackage{mathtools} % for equations
  \usepackage{physics}


  \usepackage{float}  %deactivate this if hyperref does not work with algo
  \usepackage{hyperref}
  \usepackage{tocbibind}  
  \usepackage[linesnumbered,ruled,vlined]{algorithm2e}
  \usepackage{titlesec} %for deeper subsections and changing chpt sec style
  \setcounter{secnumdepth}{4} %setting subsection 4 section deep
  % numbering equations
  \ifwhole
    \numberwithin{equation}{chapter} %equation number based on chapter
  \else
%    \numberwithin{equation}{section} % equation number based on section
  \fi

  \usepackage{xfrac} % for fraction in the for n/d
  \usepackage{esint} % for easy integral symbols

  %for listing algorithms from algorithm2e
  \newcommand{\listofalgorithmes}{\tocfile{\listalgorithmcfname}{loa}}
  %for footnote
  \renewcommand{\thefootnote}{\fnsymbol{footnote}}

  % for formattgin algorithm2e box [iman] I have to put after listofalgorithmes
  \newcommand\mycommfont[1]{\footnotesize\ttfamily\textcolor{blue}{#1}}
  \SetCommentSty{mycommfont}
  \SetKwInput{KwInput}{Input}                % Set the Input
  \SetKwInput{KwOutput}{Output}              % set the Output

  % This command is usefule when importing subdirectory items
  %margin
\usepackage{setspace}
\usepackage[
  a4paper,
  left=3.8cm,
  right=2.5cm,
  top=2.5cm,
  bottom=3cm,
  footskip=1.7cm
]
{geometry}

%IIUM font size
%syntax {\TITLEfontsize IIUM idiotize thesis formatting }
\newcommand\TITLEfontsize{\fontsize{17pt}{20.4pt}\selectfont}
\newcommand\CHAPTERfontsize{\fontsize{14pt}{16pt}\selectfont}
%For line over line for signature
\usepackage{calc}
\newcommand{\sigline}[1]{\makebox[\widthof{#1~}]{.\dotfill}\\#1}
%another over line signature command
%syantax \sign{The signant name} 
%syntax \Date but use it within minipage
%\begin{minipage}[t]{0.4\linewidth}
%    \raggedright
%    \sign{Supervisor}
%    \par
%    Mr.\,L. L. Silva\par
%    Department of Computing and Information Systems, \par
%    Faculty of Applied Sciences, \par
%    University of Moratuwa
%  \end{minipage}%
% \hfill
%  \begin{minipage}[t]{0.4\linewidth}
%    \sign{Signature of the supervisor}
%    \Date
%  \end{minipage}
\newcommand{\sign}[1]{%      
  \begin{tabular}[t]{@{}c@{}}
  \makebox[1.5in]{\dotfill}\\
  \strut#1\strut
  \end{tabular}%
}
\newcommand{\Date}{%
  \begin{tabular}[t]{@{}c@{}}
    \makebox[0.85in]{\dotfill}\\
    \strut Date \strut
  \end{tabular}%
}

%paragraph
\parindent=12mm


%header style
\usepackage{fancyhdr}
%\cfoot{\thepage}
\pagestyle{plain}
%\fancyhead{}
%\fancyhead[C]{\nouppercase{\textit{\leftmark}}} %puts chapter title on even page in lower-case italics
%\fancyhead[CO]{\nouppercase{\textit{\rightmark}}} %puts section title on odd page in lower-case italics
%\renewcommand{\headrulewidth}{0pt} %gets rid of line
\renewcommand{\chaptermark}[1]{\markboth{#1}{}} %gets rid of chapter number
\renewcommand{\sectionmark}[1]{\markright{#1}} %gets rid of section number

%making sure table and figures caption is 1 single spacecaption is 
%make sure that the tables are not in \center environment
%it will add more space between caption and the table/figures
\usepackage{caption}
\captionsetup[table]{skip=12pt}

%adding frames to a page for copygright
\usepackage{mdframed}
\usepackage{verbatim}

%using font close to Times New Roman
\usepackage{mathptmx}
% since mathptmx changes \mathcal symbols 
% use this patch to retained the default one
\DeclareMathAlphabet{\mathcal}{OMS}{cmsy}{m}{n}


\usepackage{titletoc}

%provide control over the appearance of table of contents
%figures and others
\usepackage[titles]{tocloft}
%adding dotted or dot leaders to chapter in table of content (toc)
%\renewcommand{\cftchapfont}{\bfseries}
%\renewcommand{\cftchappagefont}{\bfseries}
\renewcommand{\cftchapleader}{\cftdotfill{\cftdotsep}} % for chapters
% prepend CHAPTER on chapter number in toc
\renewcommand{\cftchappresnum}{CHAPTER }
% add ':' separator between chapter number and chapter title
\renewcommand{\cftchapaftersnum}{:}
\setlength{\cftchapnumwidth}{6.5em}
% making section normal font instead of bold
%adding "Table" keyword to the list
%provide abit of space so that the 'Table X.XX'
%keyword does not overlap with the table caption
\renewcommand{\cfttabpresnum}{Table\ }
\setlength{\cfttabnumwidth}{5.5em}

%adding the "Figure" keyword to the list
%provide abit of space so that the 'Figure X.XX'
%keyword does not overlap with the figure caption
\renewcommand{\cftfigpresnum}{Figure\ }
\setlength{\cftfignumwidth}{5.5em}

% Annotation is being displayed in bibliography/references
% This will suppress it
\DeclareSourcemap{
  \maps[datatype=bibtex]{
    \map{
      \step[fieldset=annote,     null]
      \step[fieldset=annotation, null]
    }
  }
}


  
  %This is for group notation
   \newcommand\thisPaperDir{/home/iman/version-control/ws_thesis/writing_papers/resampling_planning_in_dynamic_environment}
 
\begin{document}

\fi



\chapter{Literature Review}
\label{chap:lit_review}

\section{Introduction} 
\label{sec:chapter2_introduction}
In this chapter, the uncertainties in manipulators and the 
state-of-the-art approach to model and manage 
the uncertainty are elaborated and discussed.
The following sections use the terms \textit{error} 
and \textit{uncertainty} interchangeably. 

In an industrial environment, the uncertainty of an industrial robot is managed 
by isolating the manipulator into work cells. It lacks predictive behavior and 
only function in a predefined setup \parencite{Miseikis2017}. Such a system is 
fragile to unpredictable changes such as an accidental introduction to an object
in its environment. The robot system often costs more because of constant reprogramming 
if predefined changes are required. This review  explores solutions that enable the 
manipulator system to react appropriately to changes in the environment while taking 
account the uncertainty of its internal state. With these solutions
robot manipulator can work in a more efficient setting, i.e., involves less reprogramming and 
capable of working at close proximity with human operator safely.

This research classifies the uncertainty into two categories; 
uncertainty in the manipulator’s state and 
uncertainty in the manipulator’s environment. 
The uncertainty in a state is often caused by 
indirect measurements of the manipulator’s joint 
parameters because of the limitation of sensor choice 
\parencite{Jassemi-Zargani2002,Hebert2012,Rigatos2009,Janabi-Sharifi2010,Du2014}
and the characteristic of the manipulator’s joints and links 
\parencite{Lertpiriyasuwat2000,Ulrich2011,Lightcap2010,Badamchizadeh2010,Sawada2012}
The solutions to estimate a manipulator’s state involve filtering techniques. 
The uncertainty in an environment is often related to a lack of information about the environment’s state. 
The solutions to manage uncertainty in an environment are elaborate and non-trivial.

The following discussion will outline the uncertainty management 
of a manipulator by using an 
extended Kalman filter, unscented Kalman filter, and particle filter. 
The disucssion continues with the uncertainty management of the 
environment using point clouds and 
occupancy grid map and its variants. The uncertainty management
of the state of a robot and its environment are discussed 
based on the \acrlong{SLAM}. SLAM solutions are frameworks for more unified 
uncertainty management in autonomous robot. However, this
research will look into their application on robot manipulator.

\iffalse
The latter sections study the probabilistic approach 
at managing motion planning and the publications
that pioneered
it. The discussion of random sampling are further explained
as the probabilistic approach to planning algorithms
by referencing current research on sampling-based planners.
\fi
\section{Uncertainty Management Through State Estimation}\label{sec:uncertainty_state_estimation}
Filters are estimation tools used to observe the state of a system. 
For a manipulator robot, the state of the robot’s end effector or its end 
point is usually confirmed by direct measurement of the joint position. If this 
direct measurement is unavailable, the state of the robot is measured or estimated 
using observer or filters. 

Filtering techniques uses probabilistic model or framework to handle and manage the 
uncertainty of observing the state of the manipulator robot. It is often based on 
Baye’s rule where a model of the state is conditioned based on incomplete or indirect measurement. 
Here, the model of the state is often called a prior and the indirect or 
incomplete measurement is often regard as the update of a filter. 
If a prior of a filter is defined by another measurement, 
we consider the application of the filter as a data fusion technique. 
After conditioning the measurement update, the state estimation of 
the robot is ascertained by applying Baye’s rule. 
A filter often involves linearization like the case of an extended Kalman 
filter. 
They are also recursive such that the previous estimation 
is used as the new prior during successive sampling.

In this section, the use of extended Kalman filter, 
unscented Kalman filter, particle filter and their role 
at managing uncertainty for data fusion and state estimation 
are reviewed.

\subsection{Extended Kalman Filter for Manipulator's State Estimation}\label{sec:ekf_state_estimation}
\textcite{Jassemi-Zargani2002} used extended Kalman filter to fuse data 
from high resolution joint resolver. A joint resolver is a control unit 
that performs calculation of the inverse transformation of a manipulator 
from data obtained from the end effector. They use the data from two 
accelerometers to estimate the position of each joints in their manipulator 
and use these measurements in their extended kalman filter observer to 
estimate the acceleration of the end effector that has jerking motion. 
Their use of two sensors in extended Kalman filter 
is an example of data fusion technique.

\textcite{Lertpiriyasuwat2000} used extended kalman
filter to estimate the position 
of the end effector of two axis robotic arm together with the joint measurement 
for highly flexible links in real time. The uncertainty in their system is caused 
by the flexible links. They use a two-link manipulator. 
Each joint has optical encoder and the end point has a reflective 
infrared-light emitter. The deflection of each links is modeled using the deflection beam model. 
The dynamic equations of the manipulator are derived from Kane's method. 
They linearized the dynamic equation by eliminating the non-linear terms 
that involve the generalized elastic coordinates and their derivatives in the 
inertial matrix and the velocity vector. 
They coined the linearization method as a 'ruthless linearization'. 
The extended kalman filter is used to estimate the position of 
the end effector using the differential equation solution of 
the dynamic equation which was solved using Runge-Kutta method. 
Their result shows that with ruthlessly linearized model, 
the extended Kalman filter can estimate the position of 
the end effector consistently during low-speed and 
high-speed slew maneuvers compared to continuously linearized 
model of the dynamic model. 
From this research, linearization of model before the use of 
extended Kalman filter affects the performance of the filtering technique.

\textcite{Ulrich2011} proposed the use of extended Kalman filter to 
estimate joint positions and velocities for flexible joint positions 
and velocities for flexible joint space robotic manipulator. 
The joint flexibility in manipulators for space robots are 
obvious because they are lightweight which introduce uncertainty 
in estimating joint position. 
Thus, they extend the design of extended Kalman filters based on 
nonlinear joint models for use with an adaptive controller.
By using this combination, they increase the accuracy of 
the closed-loop estimation and control
of a flexible joint space robot.I observe that the uncertainty of 
their joints also comes from
the linearization of a nonlinear behavior of the flexible 
joint of their space robots. Although
nonlinear joints can be approximated by representing joint 
flexibility by a linear spring model, the
researchers argue that such assumption is inaccurate. 
Accordingly, they add nonlinear stiffness,
soft-windup, frictional loses, inertial cross-coupling 
to their joint model. They presented a
converging error for a non-linear based extended Kalman 
filter technique used with linear and
nonlinear joint model. However, the error diverges for 
linear based extended Kalman filter
techniques with the same model coupling.

\textcite{Lightcap2010} attempted to estimate the configuration of a
\acrfull{RLFJ} manipulator using discrete-time extended Kalman filter as 
an observer for the robot
model and control system. Because of the flexible joints, 
direct measurement from the encoders
cannot represent the position of the joint directly making the 
position of the robot’s end effector
nonlinear. The authors use extended Kalman filter to estimate the 
pose of the end effector and
manage the uncertainty of the flexible joint through linearization. 
The uncertainties of the link
and the motors' dynamics are modeled into the manipulator's dynamic 
equations. The authors
stress that extended Kalman filter has non-optimal estimation as is the 
case of any algorithm
that requires linearization. They performed a simulation and reiterate 
their models and algorithm
on a Mitsubishi PA10-6CE manipulator experimentally. 
They observed improved tracking
performance for highly flexible joints (joints under high torque) 
and low tracking performance for
rigid joints. 
Thus, the authors introduce a mixed rigid-joint/flexible joint 
model to an \acrfull{EKF-RLFJ}
controller which improved the overall tracking performance.

During visual servoing of a manipulator, the pose of the 
end effector is determined from
an intial estimate of its position. \textcite{Janabi-Sharifi2010} 
demonstrate an \acrfull{AIEKF} in the absence of accurate 
initial pose, noise matrix,
and covariance matrix at variant sampling rate during robotic visual 
servoing. Visual servoing is
a process of estimating the configuration of a manipulator using images 
or visual feedbacks.
They experimented on a six degree of freedom cartesian manipulator, 
AFMA-6, with eye-in-
hand camera configuration. 
The robot moves at a predefined trajectory under different condition.
Each separate experimentation involves increasing the velocity of 
the end effector, changing the
covariance matrices, changing the sampling time for estimation, 
and changing the initial
positions of the end effector. 
They compared three other Kalman filtering techniques (extended
Kalman filter, adaptive extended Kalman filter, and iterative 
extended Kalman filter) with the
AIEKF. They conclude that AIEKF can improve pose estimation during 
uncertain initial position,
uncertain covariance matrix estimation, high motion and slow sampling rate.

\subsection{Unscented Kalman Filter for Manipulator's State Estimation}\label{sec:ukf_state_estimation}
\textcite{Haghighipanah2015} address nonlinearity problem of an elastic cable as power
transmission between a motor and a joint for a surgical arm. The coupling between the motor
and the robot joint reduce armature mass, inertia and size for expert surgery but the cableelasticity introduces uncertainty and nonlinearity to the kinematics and the dynamics of the
surgical arm. They introduce an estimation method that uses an \acrfull{UKF} and a \acrfull{srUKF} to estimate cable coupling parameter and the
position of the end effector (end point) in real-time. The authors implement their method using
the Raven-II, a seven degree of freedom serial manipulator, as their surgical arm. The surgical
arm is equipped with an optical encoder attached to the motors and position sensor on each
joint for data validation. In their research, the authors only address the first three joint for state
estimation of the Raven-II. They model the armature dynamics using forward and inverse
dynamics with Newton-Euler equations. Their inverse dynamic solution is based on the
recursive Newton-Euler algorithm. The Newton-Euler equations requires model parameters. The
authors identify the initial inertial matrices, mass and the center of mass of the system using a
computer aided design model and use srUKF to estimate the joint angle and joint position of the
surgical arm online. Also, the authors apply the standard UKF to estimate the coupling
parameter offline. They compute spring constant, damping constant, coulomb and viscous
friction of the motor side and the joint side empirically. They validate their method
experimentally using three different design. The first two experiments involve changing the
cable tension and the third experiment involve picking an object of the mass 100 g under high
cable tension. They compared the result of the three experiment between the dynamic model
that uses their UKF estimation and the dynamic model that has no filtering technique. They
improved the accuracy of the joints position to 1.433%, 36.76%, and 62.99% for joints 1, 2 and
3 respectively. The authors also observe that whe the flexible joint are model as a rigid body,
the performance of their unscented Kalman filter deteriorate. They repeat the same problem mix
rigid-flexible cable model and replace the encoder with stereo camera \parencite{Haghighipanah2016}. 
They improve the accuracy of the joint position to 43.14\%, 33.42\%, 72.05\% for joints 1,
2, 3 respectively.

\subsection{Particle Filter for Manipulator's State Estimation}\label{sec:particle_filter_state_estimation}

\textcite{Badamchizadeh2010} present the design of two filtering techniques to estimate the
acceleration and the jerk of five-bar linkage manipulator with flexible joints. These estimations
help in designing a better control. Unlike a rigid joint, the flexible joints are time variant hence
highly nonlinear. The nonlinearity introduce uncertainty to the dynamics of the manipulator
making prediction of higher order dynamics such as accelerations and jerks uncertain. This is
because a nonlinear system has no closed form solution and require linearization. The authors
suggest the use of Euler-Langrange equations to represent the model of the manipulator
dynamics given that each joint are model by torsional spring. To better estimate the acceleration
and jerk of each link in the four-bar linkage, the linearization is manage using extended and
unscented Kalman filters. The author validated their observer model by simulation and conclude
that the accelaration and jerk of the manipulator's linkages are successfully estimated.

\textcite{Rigatos2009} uses particle filter to fuse data from an \acrfull{IMU} and joint encoders to
estimating pose of the end effector. The purpose of this paper is to estimate the state vector of a
three degree of freedom industrial robot using accelerometer and an encoder for each joint. The
estimated state vector is used to generate appropriate control signal for the manipulators.
Accelerometer are notorious for being nonlinear. The authors
use particle filter to perform data fusion between an accelerometer and an encoder to estimate
the pose of the end effector. Readers should note that both accelerometers and encoders has
non-gaussian behavior. This behavior is stronger for flexible joints because they introduce
nonlinearity to the state (position of the end effector) of the manipulator. Particle filter is used
simply because it is nonparametric. This mean that the parameter of a normal distribution are
not assumed. Instead, particle filter performs the estimation based on the sampled data and
generate the distribution from these samples. The method of sampling provide a general
solution that has no presumption of sensor characteristics. This also means that with particle
filter, more accessible sensors such as an accelerometer or an IMU can be used without scaling
down accuracy. The state vector estimation was compared with extended kalman filtering
technique. The authors observe higher accuracy in estimation of the state vector for the particle
filtering technique compared to the extended Kalman filter technique. However, the author
caution that the selection of particles numbers may improve accuracy with the expense of
computational load.

\subsection{Miscellaneous Estimation Problem in Manipulator Robot}\label{sec:misc_state_estimation}
\textcite{Du2014} use particle filter to estimate pose for a visual servoing application.
The motivation behind this paper is to introduce contactless and markerless control of a
manipulator using computer vision. The researcher uses 3D camera (Kinect sensor) to translate
a human arm motion into a motion of a manipulator and Camshift program library to track hand
position. The particle filter is used to estimate the hand position and orientation. The particle
filter handles the noise error from the Kinect sensor and the accumulated error introduced from
the Camshift method of tracking. The robot inverse kinematics are solved numerically using the
\acrfull{LM} algorithm. 

\textcite{Du2014a} use extended Kalman filter to handle kinematic errors in
manipulators. Kinematic errors occur because of imperfection in serial robot components, their
wear, misalignment and other factors. The extended Kalman filter allows auto-calibration without
strenuous technique and expensive sensors when the kinematic errors are considered. In their
method, the IMU and the position sensors are attached at the end point of the six degree of
freedom GOOGOL GRB3016 robot. The authors also explain data fusion algorithm for their IMU
attached at the end point of their serial robot where a particle filter is used to estimate the
orientation of the end point of the serial robot and a Kalman filter is used to estimate the position
of the end point of the serial robot. A Kalman filter is the base of both extended and unscented
Kalman filter which can only perform well for linear system. The extended Kalman filter is used
to optimize the position and orientation estimation of the end point of the robot. By using the
Jacobian matrices, the authors estimate the kinematic erros of the serial robot to manage the
uncertainty of using IMU measurements. \textcite{Du2014} compare their extended Kalman
filter approach to a linear least square technique for their estimation of each joints in the robot.
The extended Kalman filter has lower error for all six joint parameters estimation. The author
acclimate that their method of using IMU and position sensor, via position markers, reduce the
complex steps of auto-calibrating a manipulator, increase better accuracy, convenience, and
effectiveness.

\textcite{Hebert2012} present data fusion algorithm using unscented kalman filter to
estimate the manipulator tool and the manipulated object simultaneously. The fusion algorithm
is used to manage the uncertainty of the end effector location as a result of uncertain actuation
because of unknown weight of the manipulated object. Also, the authors use Barret WAM
manipulator that introduces further uncertainty in actuation as a result of tendon actuation similar
to flexible cable actuation. These uncertainties prompt the use of two type of sensors as
feedbacks, visual and tactile sensing. They use unscented Kalman filter to fuse image features
that covers dense range, visual appearance, silhouette of manipulator arm, multi-figered hand
and grasped object. To fuse these measurement, the authors model three measurements: (1)
the measurement model for manipulator’s hand tracking by using appearance, shape and
silhouette, (2) the object tracking measurement using point cloud association like iterative
closest points, and (3) the tactile measurement model to represent a binary state of contact
between the fingers of the manipulator and the manipulated object. The DARPA ARM-S
with Barret WAM manipulator were used to validate their methods experimentally. The experiment involves
the tasks of grasping a hand-driller and drilling a red hole on a wooden block with the grasped
drill. The authors report an average of $9.3 mm$ drilling deviation when the sensor measurements
are incorporated into the tasks and an average of $47.5 mm$ drilling deviation without the aid of
any sensor.

\textcite{Hu2017} use a new apporach, a distubance Kalman filter, to estimate the
force acting on the end effector for a compliant human-robot motion. In their approach, 
\textcite{Hu2017} modified the Kalman filtering technique by using rigid body dynamic model and its
disturbances as the update stage. The authors use \acrfull{IDM} as a prior to
the force. The approach also models the sensor using rigid body dynamic. The only feedback
data used as update state of their novel disturbance Kalman Filter are the joint positions and the
torque measurement from the sensor. The Disturbance Kalman filter takes into account the
uncertainty from the disturbance dynamics. The authors successfully implement the disturbance
Kalman filter for the force estimation on a six degree of freedom Kinova Jaco2 arm robot.

\textcite{Sawada2012} present a technique of collision avoidance using unscented
Kalman filter for a two-link flexible manipulators. The researchers use sliding mode controller to
control the motion of the manipulator. They also investigate the use of \acrshort{EKF} to
manage the uncertainty introduced by the flexible beams of the manipulator which affects the
motion trajectory of the manipulator's end effector. They introduce a collision input into the
observation model of the UKF. The collision is detected by a piezoelectric sensor attach at the
base of the links. An abnormal reading from the piezoelectric sensors would trigger the collision
input and change the parameters of the UKF to suspend control of the manipulator. This
approach, regrettably, expects collision rather than avoiding it. However, this is inevitable
considering piezoelectric sensors require contact to detect changes. They confirmed that,
through two numerical simulation, the algorithm for their collision detection via UKF successfully
increase efficiency.


\section{Map-Building Approach to Uncertainty Management}\label{sec:map_uncertainty}

A robot in an industrial environment is isolated. The robot is barricaded because of its massive
built which impose danger to human operator. But it is argued that with flexible workcell where a
robot where a robot manipulator in the industry can manage the uncertainty of its environment
through map-learning or map-building, this robot system can potentially reduce cost of
operation.

I discover that there are two methods to mapping an unknown environment of a
manipulator. The first method is by using point clouds \textcite{Um2013, Wang2017}. and
the other by using occupancy grid map together with its variants such as octomap.I must
mention that most of the researches on manipulator’s mapping are not in immediate industrial
setting. Yet,I are confident that these solutions can be used in industrial robot.

\subsection{Point Clouds Mapping}\label{sec:pointclouds_map}
Mapping with point clouds use direct data from sensor to model the environment. Often these
metrical data are translated and combined from multiple scans by using iterative closest point to
construct the global representation of the environment.

\textcite{Um2013} uses point clouds to map the environment. The paper presents a
simultaneous planning and mapping for a three degree of freedom manipulator in an unknown
environment. The paper argue that their method handles the uncertainty of the unknown
environment by using virtual skin concept. The manipulator explores the environment using a
mono-vision \acrfull{IPA} sensor which replicate the characteristic of a tactile-
sensitive skin. From this exploration, the robot would model its configuration space
and use it as the input for their best-next-move algorithm to optimize path planning.The IPA
sensors are attached on each links of the manipulator. Readers should note that an IPA sensor
has limited range and noise bound output which add more uncertainty to the estimation of the
manipulators configuration space. The researchers address the noise in the sensor using \acrfull{GAF}. The pointclouds generate from the IPA go through GAF algorithm to produce a
more confident map of the configuration space. The generated map of the configuration space is used to provide
motion planning for the robot manipulator. The RRT path
planning algorithm use the map as an input to plan the motion of the manipulator. The authors
optimized the RRT algorithm with their \acrfull{BNM} algorithm. BNM is reactive and
does not use any inverse kinematic solution. They validate the method using and compare the
RRT-BNM method with RRT method by simulation and showed that the RRT-BNM method has
higher mapping efficiency. They defined mapping efficiency as percentage of map built
compared to the actual map, over the number of point clouds in the c-space. Due to the GAF
decompression method, the RTT-BNM produce map with less point clouds.
The author, however, did not repeat the RRT experimentally. 
Despite the higher efficiency of RRT-BNM path planning, the map produce does not represent an accurate geometric
information of the unknown environment explored by the manipulator.I believe this is
because there is a lack of pose estimation of the manipulators link since the research disregard
the use of inverse kinematics, bayesian or non-bayesian filtering techniques to reconcile with
the path planning.

\textcite{Wang2017} present a method using point clouds to perform self-identification
using k-means clustering method and obstacle detection and obstacle avoidance. The point
cloud is used to generate the skeleton and the sphere along the skeleton. The point clouds are
sparse and noisy which introduce uncertainty to the self-identification. However, the uncertainty
is eliminated using a filter based on the density of the point cloud. The detection of the obstacles
is based on the negation of point clouds that are not a part of the skeleton and the sphere that
shape the manipulator. This segmentation is the base of their approach to obstacle avoidance.
They tested their algorithm by performing three experiments sequentially using a Baxter
manipulator robot with eye-in-hand configuration. These experiments involve the validation of
their self-identification algorithm, obstacle detection algorithm, and obstacle avoidance
algorithm.


\subsection{Occupancy Grid Map: The Map of Uncertain Environment}\label{sec:occupancy_map}

Direct use of point clouds to map an environment is inefficient and requires a considerable
amount of computational expense. This is not the case of an occupancy grid cell where the
point clouds are used to statistically parameterized the grid cell of the map. A grid cell or a voxel
is a cubic primitive that represent the environment’s geometrical characteristic.
Occupancy grid map consider the uncertainty of mapping sensor such as a sonar, a camera, or
a laser range finder to represent the environment probabilistically \parencite{Moravec1988}.
An example of the use of occupancy grid map is by \textcite{Matuszek2011} that
uses occupancy grid map in presenting an environment of a chess playing robot manipulator.
The construction of occupancy grid map often requires data fusion or sensor fusion where two
or more sensors are fused together using Bayesian rules or any filtering techniques.
\textcite{Rybski2012} use a variant occupancy grid map and handle uncertainty of the environment by
fusing measurements from a Swissranger SR4000's ranger and two Tyzx G3 EVS stereo
cameras to estimate the occupancy of a grid cell.

I,
accordingly, define this problem under the uncertainty managing the unpredictable environment
unlike \textcite{Janabi-Sharifi2010} which uses the state of the robot’s joint position as the
visual feedback of their visual servoing). Kalman filtering technique is used to help track an
object on a flat surface. The use of extended Kalman filter aside from the trajectory control of
the end point of the robot increase certainty of the pose estimation and the accuracy of thetracking. 
A camera is mounted on the end point of CRS Plus SRS-M1A robot system. The robot
is presented with a task to follow the object using the eye-in-hand camera. The visual feedback
from the camera is use with their control strategy. The extended Kalman filter is used to predict
the relative position of the object while it is moving. Since the researcher maintain a constant
velocity to the object, the tracking can use a Kalman filtering technique without linearization.
However, since the tracking of the object requires the movement of the camera, given the
nonlinearity of an inverse kinematic solution of their robot system, the use of extended Kalman
filter is justified. The filter maintains the estimation of the end effector's pose. Although the robot
manages to follow the moving object in their experimentation, the authors did not compare and
reiterate their experimentation with other estimation method.

\textcite{Burns2007} present a method of sample-based motion planning on top of an
occupancy grid map. The sampling-based planner take regard the uncertainty of sensor
measurements and incorporate the uncertainty into the occupancy grid map. Their method is
iterative; the planner considers and incomplete information about the world and constantly
update changes in their map model and configuration space. The authors use the concept of
utility and cost to weight the probability of an edge. A* algorithm is used to search the most
optimal path from randomly sampled node in the configuration space. During edge validation of
the roadmap is witheld before the A* algorithm complete its query. They validate their planner
by simulating three separate experiments, two of which, involve manipulators with ten degrees
of freedom and fourteen degrees of freedom respectively. The planner is specified to solve 50
planning queries given a start and a goal configurations. They report that building a planner on
top of a occupancy grid map increase the accuracy of completing the path from the start
configuration to the goal configuration. They conclude that, compared to PRM, their method
able to eliminate the possibility of invalid edges in their roadmap because they introduce error
model of the sensor at hand.

\textcite{Kruse1996} present a method for exploring an unknown environment using
either a mobie robot or a manipulator robot. Their method uses a planning-sensing-updating
cycle. This research uses an occupancy grid map to represent the unknown environment.
Their rating approach is similar to \textcite{Burns2007} utility and cost concept. However, they
relate their rating function with the manipulator's configuration space by introducing constraint
for fast exploration.

\textcite{Paul2011} present a path planning strategy to perform autonomous
grit-blasting as a part of bridge maintenance process. They develop a framework called
\acrfull{AXBAM}. AXBAM define the uncertainty in an
unknown environment as the measurement of information that has not been discovered in the
environment by using Shanon's entropy definition of information theory. This is an appropriate
concept to use in modeling an unknown environment since the theory help in optimizing the
exploration and path planning of a manipulator. The authors use Occupancy Grid map concept
to handle uncertainty for path planning in the unknown environment based on the entropy
definition. The authors implement ellipsoid force-field planner to plan collision free roadmap.
Another problem addressed by the authors states that there are possibilities for multiple goal
configuration and use their entropy model to arrive to a single goal configuration with the highest
information gain prediction. Despite the use of occupancy map in managing the control of their
manipulator under uncertain environment, 
the mapping model in this paper uses direct pointclouds for detailed geometric representation 
of the unknown environment. However, the author
fails to show any optimization technique when registering local scans into a global scan. We
believe without this optimization, if the pose of the end effector is uncertain, their map may have
diverging misalignment. The authors use Hokuyo Laser Range Finder (Hokuyo URG-04LX) as a
scanning sensor attached on a six degrees of freedom Denso VM-6093 manipulator arm with
eye-in-hand configuration. The 5th joint rotates to facilitate the initial scanning of the
environment before exploration commence. Based on this initial scan, the robot will start
exploring. By using AXBAM, the authors claim to reduce computation in decision making during
exploration.
\subsection{Octomap: The definitive Uncertainty Managament Dynamic Environment}\label{sec:octomap}

A number of researchers represent the environment using an Octree model to facilitate path
planning strategy \parencite{Faverjon1984,Hamada1996}. Similar to Occupancy grid map,
octree represent the environment with discrete cell that divides into smaller details. 
\textcite{Payeur1997} use octree based representation of occupancy grid map to reduce computational load
so that the grid map can be used efficiently to represent 3D scenes.

\textcite{Hornung2013} improves the use of octree probabilistically by introducing relaxed
logit function for uncertainty management in mobile robots.Octomap represents the uncertainty
in the environment and the sensor that maps the environment. Although Octomap has been
exhaustively used in mobile platform, researchers have introduce its use in robot manipulators.
By using Statistical Outlier Removal algorithm, \textcite{Miseikis2017} use Point clouds from
cameras to construct and environment based on octomap. They proceed by merging point cloud
data from two 3D cameras (Kinect Sensor) using iterative closest point. \acrfull{ICP} is often used in mobile robot to combine two scans collected at different position together.
They use forward kinematic to assist in replacing point clouds corresponding to the robot’s
chassis with cylindrical shapes. The map of the environment is modeled using octomap and
embedded with a decaying occupancy value. The decaying cost value are used to represent
danger zone, a mediary zone, and a non-danger zone in the map for a reflexive and predictive
behaviors. Building their path planning on top of these map provides a responsive motion even
if the robot workspace is populated by moving objects. The authors use six \acrfull{DOF} UR5 robot with
eye-to-hand configuration. They validate their method by simulating a predefined back and forth
motion between a start configuration and a goal configuration. The first simulation act as the
baseline or the benchmark of their path planning approach using only RRT. The second and the third experiments introduce a moving object into the
robot's workspace. Experiment 2 and experiment 3 use the reactive path planning and the
reflexive-predictive path planning approaches respectively. They conclude that although
experiment 3 performs at the shortest time, the result was not significant.



\subsection{Miscellaneous Mapping Techniques}\label{sec:misc_map}

In this section, I present papers that has unconventional way to manage the uncertainty oftheir environment. 
\textcite{Cohen2010} and \textcite{Meeussen2007} use graph theory to
their mapping technique, while \textcite{Petrovskaya2006} and 
\textcite{Koval2013} interact
directly with the environment using force sensor to localize object in the robot’s configuration
space. \textcite{Corrales2008} track a human operator in its environment. \textcite{Ruhr2012}
introduce the element of learning to help manipulate dynamic object in the environment.
\textcite{Cohen2010} present a search-based planning as an oppose to sampling-based
motion planning. They use examples of motion planning in lower dimensionality problem or
problems in low-dimensional manifold as a heuristic for motion planning in higher dimensional
manifold. From this heuristic, they define motion primitives, a predefined motion of a single joint,
and used them to minimize the cost function so that the most optimal path can be realized. This
method eliminated the multiple solution to start-to-goal configuration by selecting the most
feasible path that avoids collision which may not have the shortest path. Their searching-based
planner follows \acrfull{ARA*} search algorithm. ARA* is different from A*
algorithm because A* always aims at getting into a goal configuration at the shortest traversal.
ARA* consider a factor of the most optimal path initially which can be rectified at further path
sampling. They use occupancy grid map to decrease the intractability of their algorithm so that
the ARA* motion planner produces the most optimum solution in path planning. They manage
the uncertainty of multiple path solution by eliminating it using cost function as constraint to their
path planning algorithm. They validate their approach by simulating manipulation in a cluttered
tabletop and conclude that their approach are only optimal for three degree of freedom pose
(translational) rather than a full six degree of freedom pose (translational in x,y, and z directions
together with orientation about the x, y and z lines) They also perform the same experimentation
on a PR2 robot with seven degree of freedom manipulator.

\textcite{Meeussen2007} present an approach to generate a path planning by human
demonstration for a sensor based manipulator. A tool containing optical markers are tracked
during demonstration phase using a 3D vision sensor (a Krypton 6D Optical System). The tool is
attached to a geometrically uncertain object in a controlled environment. The use of 3D vision
sensor, and the estimation of the state of the tool and indirect estimation of force asserted by
the tool based on the state of the tool introduce uncertainty. To compromise the uncertainty, the
authors use particle filtering technique to estimate the pose and twist of the manipulated object
via tracking of the manipulating tool the force between the contacting object. The researcher
addresses their previous work on the same problem and optimize the particle filter with
topological graph called contact-state graph to predict the next best configuration of the object
being moved or contact formation. They observed that the contact-state graph reduces the
number of particles used during sampling which decrease computational load and increase
accuracy in their estimation. The paper however did not replicate their simulation in an
experimentation with an actual manipulator

\textcite{Petrovskaya2006} use particle filter to estimate the position of their end effector
and at the same time ascertain the position of an object by using tactile sensor. The sensing by
touching follows a heuristic where a person increases its confident of the shape and the position
of an object by feeling it with his or her hand at different location. From this heuristic, the authors
express their algorithm with the help of particle filtering technique. Particle filter depends on
sampling approach where the prior or a believe is increased by each successive samplling.
They modified the particle filtering technique to assist a highly sparse sparse measurement
fromtactile sensor and called their technique as \acrfull{SSPF}. In SSPF, the initial
filtering steps has the lowest resolution of accuracy and the highest uncertainty. Based on the
initial sampling, the following sampling increase increase the certainty of the position of an
object by sampling at a different position. The authors tested their algorithm by simulation and
experimentation. In the experimentation, SSPF manage to help a manipulator localize and
grasp a box at 70\% success rate, and identify and handle a door knob at 98\% success rate.

\textcite{Koval2013} present a situation where constant contact and manipulation of an
object, by pushing the object, can be used to estimate the state of the object; i. e. the position
and the orientation of the object. They term the process as contact manipulation. They argue
that with the absence of more conventional observation from sensors such as a laser range
finder and a vision camera, tactile based sensor can perform state estimation using appropriate
filtering technique. Since tactile sensor has non-gaussian characteristic and highly nonlinear,
the authors used particle filter as their estimator. The use of tactile sensor, however, introduce
uncertainty because the sensor has low spatial resolution or low manifold. This is regarded as a
low-dimensional manifold problem where the state estimation has higher dimensionality (two for
position on a plane and one for orientation). Although intuitively increasing the resolution of the
tactile sensor may decrease the spatial uncertainty, it is not the case for tactile sensing with
particle filtering technique. Thus, the authors introduce \acrfull{MPF}. An MPF
reduce the dimensionality of the state estimation by marginalizing the probability distribution of
the state of the object based on the observation from the tactile sensing and use it as a prior
estimate. To implement the MPF, the state estimation follows three steps:

\begin{enumerate}[(a)]
  \item assumption of the state of the object by evenly weighting particles
  \item action which involve pushing the object
  \item observation where the MPF use the pressure profile during pushing to estimate the
state of the object.
\end{enumerate}

They implement the algorithm using OpenRAVE simulation environment and evaluate it with a
simulated BarrettHand. They then run an experiment using Andy, a robot module developed for
Darpa ARM-S competition. They compared their result with a
\acrfull{CPF}
and improved estimation of the object state.

\textcite{Corrales2008} uses kalman filter fusion algorithm to fuse \acrfull{UWB}
technology and inertial motion capture system to estimate the motion of human operator in
industrial environment. The algorithm improves the interaction between a robot and a human
operator/user by localizing a human in a manipulator's workspace so that a cooperative
interaction can be made. Since the authors use UWB sensor to estimate the location of the
human, they use Kalman filtering technique to fuse the information coming from the UWB and
the inertial sensors. The filtering technique compensate the low data rate of the UWB and the
high error from inertial sensor. Their Kalman filter uses the global position of the UWB as the
correction step and the inertial sensor as the prediction step. Their result shows that by fusing
two measurements using kalman filter algorithm, the state estimation of the person's location is
increased in accuracy.

\textcite{Ruhr2012} This paper present solution to manipulation task which involve
opening and closing doors and drawers in any kitchen environment. Their approach involves the
management of uncertainty of door handling through learning. Within their learning model
framework, the authors use 3D point clouds directly to identify door or cabinet handles. They use \acrfull{RANSAC} with the point clouds to segments the point clouds to help detect the handles
based on identification of planes that parameterized a wall, ceiling, floor, and cabinets. They
also implement real-time impedance control and kinematic model learning to estimate the
kinematics of dynamic objects in the kitchen. They use a higher level of abstraction in
representing the environment via semantic maps.
They evaluated their approach using PR2 mobile articulated robot which has seven DOF manipulator.
They report out of 104 trials of opening and closing cabinets and doors, the rate of success is
51.9\%.

\section{Simultaneous Localization and Map-Building as a Total Solution to Uncertainty in a Manipulator's State and Environment}\label{sec:slam_solution}

I have discussed the uncertainty of a manipulators can be represented and manage using
filters. I have also discerned the use of occupancy grid map, point clouds, and octomap to
model the environment of a manipulator. Despite rich solution option to uncertainty of a robot
state and its environment, the solutions are disjoint and performed separately.

 

The closest solution for SLAM problem in a manipulator robot is reported by
\textcite{Klingensmith2016}. The authors argue that encoders on each joints of a manipulator is
not enough to estimate the endpoint of a manipulator due to gear trash, cable stretch non-rigid
deformation and others. Their \acrfull{ARM-SLAM} uses 
\acrfull{TSDF} as part of the scene
reconstruction to help estimate the pose of the end effector. TSDF is a variant of Dense Fusion
that performs 3D scene reconstruction using multiple depth images and camera poses. With
TSDF, each voxel is encoded with a distance to the nearest surface. The voxels are weighted
where positive weight means the voxel is outside the surface, negative weight means the voxel
is inside the surface and zero distance is when the voxel is on the surface. Their ARM-SLAM
adopt eye-in-hand configuration.They conducted three experiments to validate their SLAM
solution. In the two-dimensional simulation, they compare pose error between forward
kinematics, Dense Fusion algorithm and ARM-SLAM algorithm. The result shows significant
error reduction for both of the two algorithms compared to the forward kinematics calculation.
ARM-SLAM, however, has the lowest errors. In their 3D simulation experiment, the authors
compare results for forward kinematics calculation with kinect fusion and ARM-SLAM. They
observed that ARM-SLAM is more robust when loss of data occurs. They also conducted a real
shelf scanning. In this experiment, the authors could not conclusively see better pose estimation
compared to the forward kinematics calculation. However, they restate that during loss of data,
the ARM-SLAM solution produces robust estimation.

Based on \textcite{Klingensmith2016}, I conclude a SLAM approach represent the total uncertainty of a manipulator in an environment.
 I characterize the SLAM problem as the methodology that involve the consideration of robot’s
state uncertainty, its environment uncertainty, and a tractable solution that uses both the formerand the latter.
In a manipulator, a SLAM problem can be considered as a model that has tractable solution
given the uncertainty of its end effector’s pose and the uncertainty of its configuration space. We
define that the SLAM solution to a manipulator should have an element of map-learning where
statistical and probabilistic approach is considered when estimating the pose of an end effector.

\textcite{Sun2016} use \acrfull{LSD-SLAM} technique to replicate
the motion of a human operator. The technique resolves the noise from their measurements
using dbscan, a density-based spatial clustering algorithm, to eliminate outlier. LSD-SLAM uses
a visual camera to produce point clouds with the help of dbscan algorithm. The scenes
generated from the camera are three dimensional. However, the movement of the manipulator
is planar. The LSD-SLAM is used to replicate the movement of a human arm. The movement
are learned and modeled using \acrfull{GMM} and the parameters of the
model are estimated using \acrfull{GMR}. GMM is a probabilistic model
that assume all the data points are generated from a mixture of finite number of Gaussian
distribution with unknown parameters. GMR uses Expectation maximization iterative learning
algorithm to help replicate an output data based on an input. I observe that, by using GMM
and GMR, the authors manage to handle the movement uncertainty of the human operator and
map the movement into a planar actuation of the manipulator. The authors perform an
experiment by recording the movement of the human operator. The LSD-SLAM manage to
generate a smooth path for the end effector of the manipulator as prescribed by the GMM and
GMR based on the demonstration of the operator.

\textcite{Nissler2016} solution to AprilTag tracking is similar to SLAM solution method
where AprilTag is used as a feature observed by the measurement to optimize the pose
estimation of the end effector and at the same time using the pose estimation to track the
AprilTag on a work space. \textcite{Nissler2016} demonstrate the use of fully autonomous
handling of CFRP material. The authors use RANSAC solution to fuse multiple scene from
sensors, an eye-in-hand camera, that may contain outliers. The authors use KUKA KR 210 with
a AVT GigE camera attached to the end point. The robot follows three different motion; a
horizontal arc, a vertical arc, and an approach motion. A laser tracker is used to calibrate the
position of the robot and the markers. A single AprilTag marker was tracked during the motion.
The motions were repeated while tracking multiple AprilTag markers. RANSAC method was
compared with a least square method. They observed that RANSAC method that they
developed for the AprilTag tracking improves the estimation of the end effectors pose.

I also noted that SLAM solution is used for mobile manipulators. \textcite{Song2013}. This
paper present two problems and two solutions. One is on self-localization, a SLAM problem
which they solved using EKF-SLAM and the other is on grasping problem. I will only regard
the grasping problem since it directly involves manipulation under uncertainty. They use \acrfull{SURF} for object identification and visual servoing using Position-based
Visual Servoing (PBVS) to grasp an object and return the object to a person. During object
identification, the authors raised the concern of matching error during identification which may
result in misleading object recognition. They mitigate the problem using random sample
RANSAC to reject outliers and to find the homography matrix.
Homography matrix transforms points in one image to the corresponding points in another
image when a camera changes in position. The result of their investigation 
suggests asuccessful object retrieval to a person. They did not, however, present any comparison of their
result with other approach for the grasping problem.

From these papers, I observe that the having to mount a manipulator to a mobile
platform introduce localization uncertainty where there is uncertainty on the base location of the
manipulator arm, hence propagating the uncertainty to the path planning of the manipulator.

\textcite{Pilania2015} propose a method of base uncertainty 
by using their novel \acrfull{HAMP-U} to account for
the uncertainty of the base of a mobile robot

I also observe manipulator attached to a mobile platform often regard their grasping
solution using two different approach; i.e. a probabilisitic approach for the mobile platform
localization and a deterministic approach at the manipulation stage of the robot movement such
as grasping \parencite{Venator2013, Gasparri2006}. I also note that the total uncertainty
solution presented by SLAM does not extend to path planning approach.

\section{The Planning for 6R Robots in Dynamic Environment}

This section delves into a small number of research papers 
into motion planning in a dynamic environment for robot manipulators in three
dimensional space, $\mathbb{R}^3$. We pay close attention to 
algorithms derived from sampling-based planner which use stochastic
approach to query the configuration space. 
The planning algorithms for robot motion in a dynamics environment
dated back to 1985, albeit mainly applied on
mobile robot \parencite{Mohanan2018}. 

Some planning algorithms that is non-probabilistic 
for robotic arm 
are also observed which are
based on 
representation space \parencite{Su2011,Liu2016a}, 
sequential expanded Langrangian homotopy \parencite{Dharmawan2018a}, and
\acrfull{RAMP} \parencite{Vannoy2008}. These 
works provide a new method of planning and tackle
the problem at the local planning and at lower level of control to 
solve problems with the aid of a simulated environment. 
Their simulated
environment, or a map model of the environment, includes dynamic objects
that are then filtered and disregarded in the map registration pipeline.
Their experimentation approach will be adopted in this 
research where, the map is informed
with the presence of obstacles that are moving in the workspace without
having another sub-module of the system tracking the object via 
motion tracking.


%cite Kunz2016
%cite
\subsection{The Probabilistic Motion Planning}
\textcite{Kavraki1996} is the first group of researchers 
that used probability model for sampling 
the configuration space for holonomic robot motion 
such as a manipulator robot. The planner are called 
\acrfull{PRM} motion planning. The algorithm
construct a graph structure 
to find path between an initial pose to a goal pose in
two-dimensional configuration space, $n=2$. \textcite{Kavraki1996} also
proof a more general solution for higher dimensional configuration space,
$n>2$. With graph structure,
more than one path connect the initial pose to the goal pose. Therefore, PRM is a 
multi-query type planner. 

\textcite{Kunz2010a} improve PRM by redefining the distance metric 
of a robot manipulator so that the robot can move
around a moving obstacle in real-time.
Their approach performs well in an uncluttered environment. 
\textcite{Kunz2010a} also redefined 
the distance function of the PRM to address dynamic objects, such as
a walking person, into a two-dimensional map. Although the 
configuration space of the manipulator is in $\mathbb{R}^3$, the map
, constructed from a two-dimensional \acrfull{LiDAR} scan, is in $\mathbb{R}^2$.

%RRT introduction here
In retrospect, the RRT was formulated for 
non-holonomic motion \parencite{LaValle1998} targeting
problems addressed in diffential-constrained motion such
as a car on a plane. However, given the model of its metric space
and consequently the configuration space, RRT are tractable
for higher dimensional problem such as manipulator motion in 
3D space \parencite{Wei2018}. RRT assume as static
environment but \textcite{Wei2018} successfully 
change the way RRT samples a robot metric space so that
it is fast enough to react with a changing environment. Also, unlike PRM,
RRT works well in a cluttered environment because of 
the randomized sampling on the robot configuration space in 
the metric space. 

Researchers have been modifying the PRM 
\parencite{Klasing2007,Likhachev2005,Jaillet2004, Pomarlan2013a} 
and the RRT
\parencite{Otte2015,Ferguson2007,Ferguson2006,Bekris2007} to facilitate
better performance. 
Unlike \textcite{Kunz2010a} and \textcite{Wei2018}, so few have applied their planning
algorihtms on a robot manipulator despite both algorithms
provides mathematical framework for .


We will use the method demonstrated by \textcite{Kunz2010a} and
\textcite{Wei2018}
to design our experiment of a moving obstacle collision avoidance with
the implementation of the vanilla RRT to solve motion for robot
manipulator in three-dimensional space, $\mathbb{R}^3$. Different
from the implementation by \textcite{Wei2018} our method
implement the vanilla
RRT where we do not represent the obstacle configuration space.



\section{Summary}

It is  concluded that uncertainty management of a manipulator is imperative for a flexible
system in an industrial environment. The uncertainty of the manipulator is best
approached by considering the uncertainty motion of the robot and the uncertainty of its
environment. This thesis has reviewed research and works that manage the uncertainty of 
both categories separately. A number of researches that incorporate both uncertainties into one framework by
using simultaneous localization and mapping applied on robot manipulator is small in number. 
All of these uncertainties management involve
the use of statistical and probabilistic approach. However, no definitive Planner-SLAM solution 
has been discovered thus far in this review. Refering back to the problem statement of this 
thesis, a tractable solution that incoporate the uncertainty of the robot's state, and the
dynamic environment via a Planner-SLAM solution remains unreported. Accordingly, 
this thesis will attempt
to bridge the gap for a reconciled planner-SLAM solution into a pipeline in the following chapters.

\ifwhole\else
	\printbibliography
	\end{document}
\fi


