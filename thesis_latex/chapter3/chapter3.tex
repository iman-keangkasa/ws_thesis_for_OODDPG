\providecommand{\setflag}{\newif \ifwhole \wholefalse}
\setflag %create new if variable name ifwhole(whole)  
\ifwhole\else %if whole flag is set to true do nothing else do codes below

  \documentclass[12pt,a4paper,oneside,hidelinks]{book}

    %\input{extra_package.tex}
    %\input{tweak.tex}
    %\input{commando.tex}
    %\input{font}

%  \newcommand\thesisDir{/home/iman/version-control/ws_thesis/thesis_latex}
%  \newcommand\thisPaperDir{/home/iman/version-control/ws_thesis/writing_papers/resampling_planning_in_dynamic_environment}

  \newcommand\thesisDir{/home/asl/version-control/ws_thesis/thesis_latex}
  \newcommand\thisPaperDir{/home/asl/version-control/ws_thesis/writing_papers/resampling_planning_in_dynamic_environment}
  
  %for arabic abstract
  \usepackage[bidi=default, english]{babel}
  \usepackage[LAE, T1]{fontenc}
  \babelprovide{arabic}
  \addto\extrasarabic{\fontencoding{LAE}\selectfont}
  \addto\noextrasarabic{\fontencoding{T1}\selectfont}
  %\usepackage[LAE]{fontenc} %Not needed due to [arabic] option of the babel package
  %\usepackage[arabic]{babel}


  \usepackage{tabularx}
  \usepackage{import}
  %\usepackage{natbib}
  \usepackage[utf8]{inputenc}
  \usepackage[backend=biber,bibencoding=utf8,style=apa]{biblatex}
  \addbibresource{\thesisDir/bibtex/library.bib}
	\usepackage{soul,color}
  \usepackage{pdflscape} %for changing page orientation
  \usepackage[shortlabels]{enumitem}
  \usepackage{amssymb}
  \usepackage{graphicx} %for inserting photo
  \usepackage{subcaption}
  \usepackage{amsmath} % for equations
  \DeclareMathOperator*{\argmax}{argmax} % for underword argmax

  \usepackage{mathtools} % for equations
  \usepackage{physics}
  \usepackage{arydshln} %adding dash line to matrix


  \usepackage{float}  %deactivate this if hyperref does not work with algo
  \usepackage{hyperref}

  %to avoid self referencing TABLE OF CONTENTS 
  %use nottoc option
  \usepackage[nottoc]{tocbibind}  

  \usepackage[linesnumbered,ruled,vlined]{algorithm2e}
  \usepackage{multirow} %for merging row in cell
  \usepackage{xfrac} % for fraction in the for n/d
  \usepackage{esint} % for easy integral symbols
%chapter/section style
\usepackage[explicit]{titlesec} %for deeper subsections and changing chpt sec style
\titleformat{\chapter}
   [display] %possible values are hang, leftmargin, drop, etc.
   {\fontsize{14}{16}\bfseries\centering} % format to be applied to the whole title
   {\MakeUppercase{\chaptertitlename} \thechapter} %label setting
   {24pt} %horizontal separation between label and title body
   {\MakeUppercase{#1}} %code preceding the title body
   \titlespacing{\chapter}{0pt}{0pt}{48pt} %4spacing after title

%section style 
\titleformat{\section}
  {\normalfont\fontsize{12}{15}\bfseries}{\thesection}{1em}{\MakeUppercase{#1}}

%subsection style
\titleformat{\subsection}
  {\normalfont\fontsize{12}{15}\bfseries}{\thesubsection}{1em}{#1}

\setcounter{secnumdepth}{4} %setting subsection 4 section deep
% numbering equations
\ifwhole
  \numberwithin{equation}{chapter} %equation number based on chapter
\else
%    \numberwithin{equation}{section} % equation number based on section
\fi


  %for listing algorithms from algorithm2e
  \newcommand{\listofalgorithmes}{\tocfile{\listalgorithmcfname}{loa}}
  %for footnote
  \renewcommand{\thefootnote}{\fnsymbol{footnote}}

  % for formattgin algorithm2e box [iman] I have to put after listofalgorithmes
  \newcommand\mycommfont[1]{\footnotesize\ttfamily\textcolor{blue}{#1}}
  \SetCommentSty{mycommfont}
  \SetKwInput{KwInput}{Input}                % Set the Input
  \SetKwInput{KwOutput}{Output}              % set the Output

  
  % This command is useful when importing subdirectory items
  %margin
\usepackage{setspace}
\usepackage[
  a4paper,
  left=3.8cm,
  right=2.5cm,
  top=2.5cm,
  bottom=3cm,
  footskip=1.7cm
]
{geometry}

%IIUM font size
%syntax {\TITLEfontsize IIUM idiotize thesis formatting }
\newcommand\TITLEfontsize{\fontsize{17pt}{20.4pt}\selectfont}
\newcommand\CHAPTERfontsize{\fontsize{14pt}{16pt}\selectfont}
%For line over line for signature
\usepackage{calc}
\newcommand{\sigline}[1]{\makebox[\widthof{#1~}]{.\dotfill}\\#1}
%another over line signature command
%syantax \sign{The signant name} 
%syntax \Date but use it within minipage
%\begin{minipage}[t]{0.4\linewidth}
%    \raggedright
%    \sign{Supervisor}
%    \par
%    Mr.\,L. L. Silva\par
%    Department of Computing and Information Systems, \par
%    Faculty of Applied Sciences, \par
%    University of Moratuwa
%  \end{minipage}%
% \hfill
%  \begin{minipage}[t]{0.4\linewidth}
%    \sign{Signature of the supervisor}
%    \Date
%  \end{minipage}
\newcommand{\sign}[1]{%      
  \begin{tabular}[t]{@{}c@{}}
  \makebox[1.5in]{\dotfill}\\
  \strut#1\strut
  \end{tabular}%
}
\newcommand{\Date}{%
  \begin{tabular}[t]{@{}c@{}}
    \makebox[0.85in]{\dotfill}\\
    \strut Date \strut
  \end{tabular}%
}

%paragraph
\parindent=12mm


%header style
\usepackage{fancyhdr}
%\cfoot{\thepage}
\pagestyle{plain}
%\fancyhead{}
%\fancyhead[C]{\nouppercase{\textit{\leftmark}}} %puts chapter title on even page in lower-case italics
%\fancyhead[CO]{\nouppercase{\textit{\rightmark}}} %puts section title on odd page in lower-case italics
%\renewcommand{\headrulewidth}{0pt} %gets rid of line
\renewcommand{\chaptermark}[1]{\markboth{#1}{}} %gets rid of chapter number
\renewcommand{\sectionmark}[1]{\markright{#1}} %gets rid of section number

%making sure table and figures caption is 1 single spacecaption is 
%make sure that the tables are not in \center environment
%it will add more space between caption and the table/figures
\usepackage{caption}
\captionsetup[table]{skip=12pt}

%adding frames to a page for copygright
\usepackage{mdframed}
\usepackage{verbatim}

%using font close to Times New Roman
\usepackage{mathptmx}
% since mathptmx changes \mathcal symbols 
% use this patch to retained the default one
\DeclareMathAlphabet{\mathcal}{OMS}{cmsy}{m}{n}


\usepackage{titletoc}

%provide control over the appearance of table of contents
%figures and others
\usepackage[titles]{tocloft}
%adding dotted or dot leaders to chapter in table of content (toc)
%\renewcommand{\cftchapfont}{\bfseries}
%\renewcommand{\cftchappagefont}{\bfseries}
\renewcommand{\cftchapleader}{\cftdotfill{\cftdotsep}} % for chapters
% prepend CHAPTER on chapter number in toc
\renewcommand{\cftchappresnum}{CHAPTER }
% add ':' separator between chapter number and chapter title
\renewcommand{\cftchapaftersnum}{:}
\setlength{\cftchapnumwidth}{6.5em}
% making section normal font instead of bold
%adding "Table" keyword to the list
%provide abit of space so that the 'Table X.XX'
%keyword does not overlap with the table caption
\renewcommand{\cfttabpresnum}{Table\ }
\setlength{\cfttabnumwidth}{5.5em}

%adding the "Figure" keyword to the list
%provide abit of space so that the 'Figure X.XX'
%keyword does not overlap with the figure caption
\renewcommand{\cftfigpresnum}{Figure\ }
\setlength{\cftfignumwidth}{5.5em}

% Annotation is being displayed in bibliography/references
% This will suppress it
\DeclareSourcemap{
  \maps[datatype=bibtex]{
    \map{
      \step[fieldset=annote,     null]
      \step[fieldset=annotation, null]
    }
  }
}


  \usepackage[acronym,symbols,nogroupskip,nonumberlist]{glossaries-extra}
%\usepackage[toc,acronym]{glossaries}
%list of abbreviation
% in the main change the title of the abbreviation using 
% \printglossary[type=\acronymtype,title={LIST OF ABBREVIATIONS},titletoc={List Of Abbreviations]
% use \acrlong{SLAM} to print Simultaneous Loc. And Mapping
% use \acrshort{SLAM} to print SLAM
% use \acrfull{SLAM} to print Simulataneous Localization and Mapping (SLAM) 
% to add entry: \newacronym{label}{the_acronym}{acro description}
%to add entry \newacronym{slam}{SLAM}{simultaneous localization and mapping}
\makeglossaries

\newacronym{SME}{SME}{Small and Medium-Sized Enterprise}
\newacronym{RRT}{RRT}{Rapidly-Exploring Random Tree}
\newacronym{RTAB-Map}{RTAB-Map}{Real-Time Appearance-Based Mapping}
\newacronym{3D}{3D}{three-dimensional}
\newacronym{FAS}{FAS}{Flexible Automation System}
\newacronym{SLAM}{SLAM}{Simultaneous Localization and Mapping}
\newacronym{RGB-D}{RGB-D}{Red-Green-Blue-Depth sensor or visual-depth camera/sensor}
\newacronym{RLFJ}{RLFJ}{Rigid Link Flexible Joint}
\newacronym{EKF}{EKF}{Extended Kalman Filter}
\newacronym{EKF-RLFJ}{EKF-RLFJ}{Extended Kalman Filter and Rigid Link Flexible Joint coupling}
\newacronym{AIEKF}{AIEKF}{Adaptive Iterative Extended Kalman Filter}
\newacronym{UKF}{UKF}{Unscented Kalman Filter}
\newacronym{srUKF}{srUKF}{Square Root Unscented Kalman Filter}
\newacronym{IMU}{IMU}{Inertial Measurement Unit}
\newacronym{LM}{LM}{Levenberg-Marquardt}
\newacronym{IDM}{IDM}{Inverse Dynamic Model}
\newacronym{IPA}{IPA}{Infrared Proximity Array}
\newacronym{GAF}{GAF}{Group Average Feature}
\newacronym{BNM}{BNM}{Best Next Move}
\newacronym{AXBAM}{AXBAM}{Autonomous eXploration to Build A Map}
\newacronym{ICP}{ICP}{Iterative Closes Point}
\newacronym{DOF}{DOF}{degree-of-freedom}
\newacronym{ARA*}{ARA*}{Anytime Repairing A*}
\newacronym{SSPF}{SSPF}{Scaling Series Particle Filter}
\newacronym{MPF}{MPF}{Manifold Particle Filter}
\newacronym{CPF}{CPF}{Conventional Particle Filter}
\newacronym{UWB}{UWB}{Ultra-wideband}
\newacronym{RANSAC}{RANSAC}{Random Consensus}
\newacronym{ARM-SLAM}{ARM-SLAM}{Articulated Robot Motion for Simultaneous Localization and Mapping}
\newacronym{TSDF}{TSDF}{Truncated Signed Distance Field}
\newacronym{LSD-SLAM}{LSD-SLAM}{Large-scale Direct mono-SLAM}
\newacronym{GMM}{GMM}{Gaussian Mixture Model}
\newacronym{GMR}{GMR}{Gaussian Mixture Regression}
\newacronym{SURF}{SURF}{Speed Up Robust Feature}
\newacronym{PBVS}{PBVS}{Position-based Visual Servoing}
\newacronym{HAMP-U}{HAMP-U}{Hierarchical and Adaptive Mobile Manipulator Planner Uncertainty}
\newacronym{RAMP}{RAMP}{Real-time Adaptive Motion Planning}
\newacronym{PRM}{PRM}{The Probabilistic Roadmap}
\newacronym{LiDAR}{LiDAR}{Laser Imaging, Detection and Ranging}
\newacronym{rimini}{r\_mini}{Richard Mini, a compliant six-axis manipulator}
\newacronym{PHASER}{PHASER}{Phase Spherical Harmonics approach to map registration}

\glsxtrnewsymbol[sort={0_p},description={Bayes' posterior}]
{p}
{\ensuremath{p}} 

\glsxtrnewsymbol[sort={0_mi},description={Global or local map}]
{mi}
{ \ensuremath{m_{i}} }

\glsxtrnewsymbol[sort={0_SO3},description={Special orthogonal group in 3D space representing rotation and rotational algebra}]
{SO3}
{ \ensuremath{\vb*{SO(3)}} }

\glsxtrnewsymbol[sort={0_xi},description={State in 3D space}]
{xi}
{ \ensuremath{x_{i}} }

\glsxtrnewsymbol[sort={0_xihat},description={State estimation in 3D space}]
{xestimate}
{ \ensuremath{ \hat{x}_{i} } }

\glsxtrnewsymbol[sort={0_Rnsuper},description={Real n-vector}]
{Rn}
{ \ensuremath{ \mathbb{R}^{n}  } }


\glsxtrnewsymbol[sort={0_R3},description={3D space specifically refering to the Euclidean space}]
{R3}
{ \ensuremath{ \mathbb{R}^3 } }


\glsxtrnewsymbol[sort={0_SE3},description={Special Euclidean space in 3D space}]
{SE3}
{ \ensuremath{ \vb*{SE(3)}  } }

\glsxtrnewsymbol[sort={0_zi},description={Observation model}]
{zi}
{ \ensuremath{ z_{i}  } }

\glsxtrnewsymbol[sort={0_ui},description={State transition model}]
{ui}
{ \ensuremath{ u_{i}  } }

\glsxtrnewsymbol[sort={0_Cn},description={Set of configuration space on 3D, homeomorphic to the open set of constricted 6-sphere space}]
{ConfigurationSpace}
{ \ensuremath{ C_n  } }

\glsxtrnewsymbol[sort={0_Cee},description={Set of configuration space for the end-effector on $\mathbb{R}^3 \times \vb*{SO(3)}$}]
{Configurationee}
{ \ensuremath{ C_{ee}  } }

\glsxtrnewsymbol[sort={0_Cnsuper},description={Control space, specifically the n-joint robot manipulator}]
{ControlSpace}
{ \ensuremath{ C^{n} } }

\glsxtrnewsymbol[sort={0_Ceesuper},description={Control space at the end-effector or the task space}]
{Controlee}
{ \ensuremath{ C^{ee} } }

\glsxtrnewsymbol[sort={0_ceesuper},description={Element in control space at the end-effector or the task space}]
{controlee}
{ \ensuremath{ c^{ee} } }

\glsxtrnewsymbol[sort={0_cee},description={Element in configuration space at the end-effector or the task space}]
{configurationee}
{ \ensuremath{ c_{ee} } }

\glsxtrnewsymbol[sort={0_ceesuperhat},description={Estimated element in control space at the end-effector or the task space}]
{estimatecontrolee}
{ \ensuremath{ \hat{c}^{ee} } }

\glsxtrnewsymbol[sort={0_cnsuper},description={Element in the control space, $C^{n}$}]
{controlspace}
{ \ensuremath{ c^{n} } }

\glsxtrnewsymbol[sort={0_cn},description={Element in the configuration space, $C_{n}$}]
{configurationspace}
{ \ensuremath{ c_{n} } }

\glsxtrnewsymbol[sort={0_tvector},description={Translation 3-vector}]
{translationVector}
{ \ensuremath{\vb*{t}} }

\glsxtrnewsymbol[sort={0_Rvector},description={Rotation matrix}]
{rotationMatrix}
{ \ensuremath{ \vb*{R} } }

\glsxtrnewsymbol[sort={1_mui},description={Instruction vector from path and motion planner pipeline}]
{mui}
{ \ensuremath{ \mu_{i} } }

\glsxtrnewsymbol[sort={0_Ccycle},description={Subset of the configuration space defined by the cycle space generator, algorithm \ref{algo:cycle_space}}]
{Ccycle}
{ \ensuremath{ C_{cycle} } }

%use \mathsf{} for cartesian-axis related notation

\glsxtrnewsymbol[sort={0_axisz},description={The z axis on the Cartesian coordinate system}]
{zAxis}
{ \ensuremath{ \mathsf{z-axis}  } }

\glsxtrnewsymbol[sort={0_axisZYZ},description={The Z-Y-Z Euler angles}]
{zyzAngle}
{ \ensuremath{ \mathsf{Z-Y-Z}  } }

\glsxtrnewsymbol[sort={0_axisy},description={The y-component of the Cartesian coordinate system}]
{yAxis}
{ \ensuremath{ \mathsf{y-axis}  } }

\glsxtrnewsymbol[sort={0_axisx},description={The x-component of the Cartesian coordinate system}]
{xAxis}
{ \ensuremath{ \mathsf{x-axis}  } }

\glsxtrnewsymbol[sort={0_ai},description={Denavit-Hartenberg's $\mathsf{z-axis}$ offset parameter}]
{ai}
{ \ensuremath{a_i} }

\glsxtrnewsymbol[sort={0_di},description={Denavit-Hartenberg's $\mathsf{x-axis}$  offset parameter}]
{di}
{ \ensuremath{ d_i  } }

\glsxtrnewsymbol[sort={1_alphai},description={Denavit-Hartenberg's $frame_i$'s x-rotational offset parameter}]
{alphai}
{ \ensuremath{ \alpha_i  } }

\glsxtrnewsymbol[sort={0_Ai},description={Homogenous transformation of rigid body in 3D}]
{homogenousTransformation}
{ \ensuremath{ A_i  } }

\glsxtrnewsymbol[sort={0_T},description={A 4$\times$4 matrix encoded with $\vb*{t}$ translation along an axis, at an angle}]
{homogenousTranslation}
{ \ensuremath{ T_{axis,angle} } }

\glsxtrnewsymbol[sort={0_Rot},description={A 4$\times$4 matrix encoded with $\vb*{R}$ rotation along an axis, at an angle}]
{homogenousRotation}
{ \ensuremath{ Rot_{axis,angle} } }

\glsxtrnewsymbol[sort={1_deltat},description={Distance between two points in 3D}]
{deltaTranslation}
{ \ensuremath{ \Delta t } }

\glsxtrnewsymbol[sort={0_Mz},description={Collision map in \gls{mi}}]
{collisionMap}
{ \ensuremath{ \mathcal{M} } }

\glsxtrnewsymbol[sort={0_Tz},description={Tree node structure in \gls{mi}}]
{treeNode}
{ \ensuremath{ \mathcal{T} } }

\glsxtrnewsymbol[sort={0_crandom},description={Random sampling in the configuration space}]
{configurationRandom}
{ \ensuremath{ c_{random} } }

\glsxtrnewsymbol[sort={0_Climit},description={Constrained imposed on the configuration space}]
{ControlLimit}
{ \ensuremath{ C_{limit}  } }

\glsxtrnewsymbol[sort={0_q},description={The 4-vector parameterization of the quaternions in the Hamiltonian space}]
{quaternion}
{ \ensuremath{ \vb*{q} } }

\glsxtrnewsymbol[sort={0_H},description={The Hamiltonian space}]
{Hamiltonian}
{ \ensuremath{ \mathbb{H} } }

\glsxtrnewsymbol[sort={1_tau},description={Trajectory in control space, \gls{ControlSpace}}]
{trajectory}
{ \ensuremath{ \tau } }

\glsxtrnewsymbol[sort={0_Cfree},description={Non-colliding configuration space}]
{ConfigurationFree}
{ \ensuremath{ C_{free} } }

\glsxtrnewsymbol[sort={0_Cworkspace},description={Valid reachable pose constrained by the joint angles}]
{ConfigurationWorkspace}
{ \ensuremath{ C_{workspace} } }

\glsxtrnewsymbol[sort={0_Cobstacle},description={Occupied space, i.e. obstacles in the worksapce and the robot arm itself}]
{ConfigurationObstacle}
{ \ensuremath{ C_{obstacle} } }

\glsxtrnewsymbol[sort={0_ftilde},description={A target 3D pointcloud treated as a function}]
{fPointcloud}
{ \ensuremath{ \tilde{f} } }

\glsxtrnewsymbol[sort={0_gtilde},description={A 3D pointcloud treated as a function}]
{gPointcloud}
{ \ensuremath{ \tilde{g} } }

\glsxtrnewsymbol[sort={0_L2S2},description={Hilbert space and algebra on 2-sphere}]
{hilbertSpace}
{ \ensuremath{ L^2(S^2) } }

\glsxtrnewsymbol[sort={0_S2},description={The topology of a 2-sphere}]
{2Sphere}
{ \ensuremath{ S^2 } }

\glsxtrnewsymbol[sort={0_l},description={The spherical harmonics degree}]
{sphericalHarmonicsDegree}
{ \ensuremath{ l } }

\glsxtrnewsymbol[sort={0_m},description={The spherical harmonics order}]
{sphericalHarmonicsOrder}
{ \ensuremath{ m } }

\glsxtrnewsymbol[sort={0_Pml},description={The Legendre Polynomial}]
{LegendrePolynomial}
{ \ensuremath{ P_m^l } }

\glsxtrnewsymbol[sort={1_theta},description={The longitude of the 2-sphere}]
{longitudeTheta}
{ \ensuremath{ \theta } }

\glsxtrnewsymbol[sort={1_psi},description={The latitude of the 2-sphere}]
{latitudePsi}
{ \ensuremath{ \psi } }

\glsxtrnewsymbol[sort={0_Yml},description={The spherical harmonics component of the spatial Fourier transform}]
{sphericalHarmonics}
{ \ensuremath{ Y_m^l } }

\glsxtrnewsymbol[sort={0_Btilde},description={The spherical bandwidth used to discretize the surface of the 2-sphere}]
{sphericalBandwidth}
{ \ensuremath{ \tilde{B} } }

\glsxtrnewsymbol[sort={0_Ln},description={The logit function}]
{logitFunction}
{ \ensuremath{ L(n) } }

\glsxtrnewsymbol[sort={0_Pn},description={Posterior probability function for the occupancy of a voxel}]
{occupancyProbability}
{ \ensuremath{ P(n) } }

\glsxtrnewsymbol[sort={0_n},description={Node representing the discretization of 3D space called voxel}]
{voxel}
{ \ensuremath{ n } }

\glsxtrnewsymbol[sort={0_Ctilde},description={Correlation on 2-sphere}]
{Correlation}
{ \ensuremath{ \tilde{C} } }

\glsxtrnewsymbol[sort={0_D},description={The Weigner-D matrix or function}]
{WeignerD}
{ \ensuremath{ D } }

\glsxtrnewsymbol[sort={0_zzzboxplus},description={A Gaussian fusion technique}]
{GaussianFusion}
{ \ensuremath{ \boxplus } }

\glsxtrnewsymbol[sort={0_J},description={The kinematic Jacobian}]
{Jacobian}
{ \ensuremath{ J } }

\glsxtrnewsymbol[sort={1_phi},description={Map represented by the Octomap-harmonics posterior from the spherical harmonics state estimation pipeline}]
{OctomapWorkspace}
{ \ensuremath{ \Phi  } }

\glsxtrnewsymbol[sort={0_t},description={Time}]
{time}
{ \ensuremath{ t } }




  \newcommand\rimini[0]{\textit{r\_mini}} %r_mini typesetting

  %This is for group notation
  
\begin{document}

\fi



\chapter{Mathematical Backgrounds}\label{chap:mathematical_background}

%\section{Introduction}\label{sec:chapter1_introduction}
\noindent
The role of this chapter is to provide the primer
for the mathematical foundation of this thesis. 

A six-axis robot with six revolute joints has been developed
as the hardware platform to validate the Planner-SLAM
solution proposed in this thesis. The robot has been
named Richard Mini (code as \glsxtrshort{rimini}). This chapter presents
the features and mathematical model of this robot.

The chapter revolves around the geometric 
representation of the kinematic model 
of \rimini~ and the mathematical 
framework for the planner used in this research
and the components of the SLAM solution. 

%[task] follow how important_IK_Jacobian_Methods.pdf explains the kinematics
\section{The Kinematics of Richard Mini}\label{sec:kinematics_rimini}
\subsection{The Mechanical Descriptions}\label{sec:mechanical_description}

\import{\thesisDir/chapter3/figures/}{r_mini_pieper.tex} %fig:r_mini_pieper
%[task] explain rotation aspect of the kinematics 
 
Richard Mini (\rimini) has one fixed link, 
six moving links, six revolute joints, and 
six frames attached to the links. The links are named numerical:
link1, link2, \ldots, link6 shown in figure \ref{fig:rimini_links}.
The same is true for the joints: joint1,
joint2, $\ldots$, joint6. All joints are described by the frames
attached to the links. In figure \ref{fig:rimini_joints}, the frames' orientations
are illustrated. 
These joints represent the state of 
the manipulator which parameterized the control space $\{\gls{controlspace} \in \gls{ControlSpace}\}$. $ \gls{ControlSpace} $ for \rimini~ 
is a 6-dimensional control space.

This thesis
will use the superscript notation to refer the control space set and its elements and the 
subscript as the equivalent notation in the configuration space.
For example, $C^{ee}$, refers to the control space at the end-effector
where the controlling pipelines for \rimini~ would take in
$c^{ee} = (\theta_{1},\ldots,\theta_{6}) \in C^{6}$, 
and the equivalent pose is in the configuration space, $C_{ee}$.
\rimini~ follows the condition advised by \textcite{Pieper1968}
:three last joints are collated and have a shared rotation axis point as
prescribed in figure \ref{fig:r_mini_pieper}.

The figure \ref{fig:rimini_joints} shows the $\gls{zAxis}$ of 
the frames aligns with the rotation of the actuators on the robot. 

\import{\thesisDir/chapter3/figures/}{rimini_joints.tex} %fig:rimini_joints
\import{\thesisDir/chapter3/figures/}{rimini_links.tex} %fig:rimini_links

These three joints represent the wrist. 
\rimini~ also uses $\gls{zyzAngle}$ Euler convention to present rotation state
of the end effector. This convention conforms
to the Wigner-D matrix parameterization where this 
research will use to estimate the orientation of 
the end-effector. However, the forward and inverse 
kinematic computation uses the quaternion.

\subsection{Forward Kinematics of \rimini~}\label{sec:fk_rimini}
The kinematics of \rimini~ follows the Denavit-Hartenberg (DH) convention. 
In this convention, the homogenous transformation of one frame to the next frame
in the kinematic chain of the robot does not entails any rotation about the $\gls{yAxis}$.
In so doing, the homogenous transformation between two frames are simplified. By disregarding
the rotation about the $\mathsf{y-axis}$, matrix-to-matrix multiplication between two frames required 
for homogenous transformation is computationally relax. To help satisfy this constraint 
about the $\mathsf{y-axis}$, two condition should be followed:

\begin{itemize}
  \item The $\gls{xAxis}$ belonging to $frame_i$ should be perpendicular to the 
    $\mathsf{z_{i-1}-axis}$
    belonging to the $frame_{i-1}$.
  \item The $\mathsf{x_i-axis}$ intersects the $\mathsf{x_{i-1}-axis}$
\end{itemize}

Here, the subscript $\mathsf{i}$ represent the frame number which extend to the link and joint number. 
Based on these constraints, the DH parameters are summarized in table \ref{tab:rimini_dh_table},
where $\gls{ai}$ is the offset of the $frame_i$'s $\gls{zAxis}$ along the $\gls{xAxis}$, $\alpha_i$
is the rotation of $frame_i$ about the $\mathsf{x_i-axis}$, $\gls{di}$ is the offset of the
$frame_i$'s $\gls{xAxis}$ along the $\mathsf{z_i-axis}$, and $\theta_i$ is the $joint_i$ angle of rotation. 
DH-convention states that each joint values are defined by the angle between two $\mathsf{x-axes}$. As an
example joint1 is represented by the angle between frames attached to link1
and the base frame. Similarly joint2 is represented by the angle between $\mathsf{x_1}$
and $\mathsf{x_2}$ axes.


  \import{\thesisDir/chapter3/equations/}{rimini_dh_table.tex}

Each row in table \ref{tab:rimini_dh_table} describes the homogenous transformation, 
$\gls{homogenousTransformation}$ from the 
$frame_{i-1}$ to $frame_i$. The transformation is repesented by four matrix operation:

  \import{\thesisDir/chapter3/equations/}{homogenous_transformation.tex}

where $\gls{homogenousRotation}$, and $\gls{homogenousTranslation}$ are the representation of rotation and translation in special
Euclidean group $\vb*{SE(3)}$. Here $A\in\vb*{SE(3)}$. 
The homogenous transformation
is represented by the homogenouns transformation matrix defined as: 

\import{\thesisDir/chapter3/equations/}{homogenous_matrix.tex}

where $\vb*{R} \in \vb*{SO(3)}$ and $\vb*{t} \in \mathbb{R}^3$.
$\vb*{SO(3)}$ is the special orthogonal group that represent
orientation of the frames in the 3D space.

With each row of the DH-parameter table \ref{tab:rimini_dh_table} represented
by a homogenous transformation matrix $A$, \rimini~ configuration
space is mapped by the forward kinematic function $f_{FK}:\gls{ConfigurationSpace} \to \mathbb{R}^3$:

  \import{\thesisDir/chapter3/equations/}{fk_mapping.tex}

\subsection{Inverse Kinematics of \rimini}\label{sec:ik_rimini}

The inverse kinematics solution for \rimini~ follows the analytical
approach derived by \textcite{Pires2007}, since both the IRB1600 and \rimini
has the Pieper condition addressed in section \ref{sec:mechanical_description}.
However, given that the analytical approach
introduce multiple solution, accordingly, 
subjected the motion planning to singularities, this report adopts a 
numerical method approach which is 
used during the simulation and experimentation. 

Newton-Raphson method to find the inverse kinematic solution of 
\rimini, $\gls{estimatecontrolee}$ which is the estimate of $\gls{controlee}$. The generalization of the method uses the current value of the 
robot's encoder, $c^{current}$, and the termination value, $\epsilon=0.005$,
to end the iteration. Algorithm \ref{algo:newton_raphson} delineate the 
the method.

\import{\thisPaperDir/algorithms/}{newton_raphson.tex}
\import{\thisPaperDir/algorithms/}{RRT.tex}

\section{\rimini's Path Planner}\label{sec:planning_rimini}
This research uses RRT implementation provided by OMPL library
packaged in the MoveIt software. The algorithm for the purpose 
of this research is shown in \textit{algorithm \ref{algo:RRT}}
where $k$ represent the, number of node in the 
tree generated by the RRT, $\gls{collisionMap}$, represent the collision map
which is part of the planning scene where all
RRT sampling takes place and $\gls{treeNode}$ is the tree 
that points to a non-colliding space. In this RRT implementation, 
the map are loaded or queried in line \ref{ln:initialize_tree} 
each time the \textit{generateRRT()} is invoked.
Line~\ref{ln:random_generator} generates a random state bias towards
the $C_{goal}$.
Line~\ref{ln:knn} invokes the k-nearest neighbor to find a selection 
of nodes that is closes to the state configuration, $\gls{configurationRandom}$.
Line~\ref{ln:direction} is the important part of the sampling 
in the RRT where it represent the controlling input of the
robot motion. Since, the robot are control in the joint-configuration
space, the angular joint limit address the shape
of the workspace. However, given the angular velocity, 
these limits are translated into 
the configuration space via the kinematic Jacobian which requires
the information on the $\gls{deltaTranslation}$, a tree node searching constraint. 
The limits implicitly ensures 
that the RRT, by executing line~\ref{ln:direction} within the context 
of the robots Jacobian, does not pass through the singularities of 
the robot. Hence, the configuration space of the manipulator 
also includes, $\gls{ControlLimit}$, containing configuration
that abides the joint-control space ranges and angular velocity
limits.

The configuration space where 
the RRT-sampling is concerned is modified in this thesis where, 
the rotation representation and its sampling is in
$R \in \gls{Hamiltonian}$, such that the parameterization of
the Hamiltonian-space is the quaternions 4-vector,
$\gls{quaternion} \in \mathbb{R}^4$. Therefore, the representation of the robot poses
and also the non-colliding poses, $(\vb*{q},\vb*{t})$ 
are explained in \textit{equation \ref{eq:configuration_vector}}.

\import{\thisPaperDir/equations/}{configuration_vector.tex}

The RRT sampling involves query into a map, that stores collision objects.
This is the planning scene, denoted as collision map, where the RRT sampling occurs.
The query are
called when both initial pose and a goal pose are sent to the RRT planner input. 
The output of the pipeline is a set of non-colliding space where from there another
pipeline, transform the configuration space into a control space. The control space is 
definedin the following section.
  % matrix t/q of the C_pose = (t,q)
%write the algorithm here
%read section 2 of Lavelle1998
\subsection{The Cycle Space}
The cycle space is the subset of the planner solution where the RRT algorithm is invoke
twice. 
During the generation of the cycle space,  the RRT
output the a trajectory from the
initial pose, $c_{initial}$ to the goal pose, $c_{goal}$, 
into a controlling pipeline. The trajectory
are then map from the configuration space into the joint-configuration space via
the Newton-Raphson inverse kinematic solver 
(algorithm \ref{algo:newton_raphson}). To complete 
the set of the cycle space consist of trajectories, $\gls{trajectory}$, the entries in the initial pose and the goal pose
are swapped, while invoking query into the collision map,
\import{\thisPaperDir/equations/}{trajectory.tex}

which forms a cyclical motion between the initial pose and the goal
pose. Here, 
$\gls{Ccycle} = \overbrace{ C_{initial \rightarrow goal}}^\text{algorithm \ref{algo:cycle_space} line \ref{ln:cs_RRT1}} 
              + \overbrace{C_{goal \rightarrow initial}}^\text{algorithm \ref{algo:cycle_space} line \ref{ln:cs_RRT2}}$.
Algorithm~\ref{algo:cycle_space} block explains how the $C_{cycle}$ space is constructed.

\import{\thisPaperDir/algorithms/}{cycle_space.tex}



The control space are represented by the trajectory in the joint-control
space $C^{control} \subset C^{cycle}$. In \textit{equation \ref{eq:trajectory}} the
joint-configuration space are equivalent with the configuration space in 
\textit{equation \ref{eq:configuration_vector}}. 
The control space
is the direct controlling parameters for the movement of 
\rimini~ where it
only handles the control space (or joint-state space).
The sampling of the RRT to generate the tree data structure, $\mathcal{T}$,
are done within the $\vb{SO(3)} \times \mathbb{R}^3$ topology. The free
configuration space, or the non-colliding pose, are represented by, 
$\gls{ConfigurationFree} = \gls{ConfigurationWorkspace} / \gls{ConfigurationObstacle}$. 
According to 
\textcite{LaValle1998}, the $C_{obstacle}$ also covers 
the physical contraint of the non-holonomic movement of the robot. 
However, in the case of an articulated robot arm in this research
, the configuration 
limitation are the range of the joints and the angular velocity limit. 
Since, all of these
measurements are in the n-hyperspace, to map them into the
$\vb*{SO(3)} \times \mathbb{R}^3$, we use the kinematic Jacobian.

\section{The Spherical Harmonics}\label{sec:spherical_harmonics}
In this thesis, spherical harmonics are used to discern the 
estimate of the orientation of \rimini's end effector via the 
raw data coming from the RGB-D sensor. The spherical analysis derives from 
the fourier transformation of non-periodic function. Here, 
we will represent the pointclouds coming from the RGB-D sensor 
as an unknown functions \parencite{Osteen2012}. With spherical harmonics, the state of the 
end effector are computed based on two overlapping functions $\gls{fPointcloud}$
and $\gls{gPointcloud}$ both representing the pointclouds. These functions
are projected into a unit-hypersphere $\gls{2Sphere} =\{x \in \mathbb{R}^3\: | {\Vert x \Vert}_2 = 1\}$.
where the projection of $\tilde{g}$ and $\tilde{f}$ based on the
same parameterization in \textcite{Healy2003}. Convolution of the two function 
are represented by inner product, acting on the Hilbert space, $L^2(S^2)$:

\import{\thesisDir/chapter3/equations/}{hilbert_product.tex}

The implementation of the equations \ref{eq:spherical_coefficient} 
\ref{eq:fourier_coefficient} and \ref{fourier_transform} are coupled
with the datastructures of in equations \ref{eq:octomap_relax} \ref{eq:octomap_red} 
\ref{eq:octomap_green} and \ref{eq:octomap_blue}
to form a correlation between the two overlapping functions $\tilde{g} \in L^2{S^2}$ and 
$\tilde{f} \in L^2(S^2)$: 

\import{\thesisDir/chapter3/equations/}{fourier_correlation.tex}

%\section{The Probability Model of \rimini}\label{probability_rimini}

%The essence of the duality configuration on each solution of 
%the first three joints in \rimini is a probabilistic in nature. At
%a specific pose, described by both the the translation and orientation 
%of the pose encodes eight different configuration, the forms a probability
%distribution, 



%\subsection{The Cost Function of Kinematics}\label{sec:cost_function}
\section{The Use of Kinematics Jacobian and the Singularities of the Manipulator to Relax the Cost Function}\label{sec:singularity_4_cost_function}
%[task] add mapping equations from configuration space to euclidean space
%[task] add a picture that shows a near zero jacobian's determinant
%[task] add symbols
The kinematic Jacobian is defined as,
\import{\thesisDir/chapter3/equations/}{jacobian.tex}

where given the right-hand side of the equation,
angular velocity in any of the joints changes abruptly when the determinant of $J$ 
approaches zero.
This abrupt changes is not just because
of the zero-determinant of the Jacobian, $\gls{Jacobian}$, but also
a consequence of the duality in the solution explained in section \ref{sec:ik_rimini}.
At a point where the determinant of the Jacobian 
of the fundamental kinematics equations approaches to zero, 
the angular velocity of any of the joint state approaches to infinity.
This behavior is represented by the mapping of the configuration space 
to the end effector's state. %add SO(3) symbol here

%[task]equation here p = T^*omega
%[task]equation here mu = J^{-1}*omega
The fundamental kinematics equation demonstrate a peculiar characteristic.
The equation \ref{eq:jacobian} explains, regardless of the infinity velocity when approaching 
singularity, the task space does not reflect 
any abrupt or catastrophic movement. This behavior when discern in context
of kinematic equation \ref{eq:jacobian}, shows that the robot configuration 
changes abruptly when travesing into a subset of the workspace
that coincide with 
zero-determinant Jacobian.
%[task]picture the euclidean group movement vs the configuration space 
%[task]notation map function
Should a closed-form solution to the subset of the function that represent the 
singularity in the workspace is needed, readers can venture into
invoking the fundamental theory of determinants that give
constraints and conditions to satisfy non-invertible Jacobian. However, caution ensues as
this efforts may not be trivial if solved analytically. 
However, since the sampling space for the RRT includes the 
kinematic constraint and also the velocity constraint of the
robot arm, the solution from the RRT, and therefore, the trajectory of the
robot will never cross the singularity region. This
implies that, the determinant of the Jacobians, when RRT sampling 
considers the kinematic and velocity contraint of the robot, will
never approach zero. 
\section{SLAM for \rimini}\label{sec:slam_graph}

\rimini~'s end effector is an RGB-D camera: the Intel Realsense
435D. The camera livefeeds point clouds, imagery, and a calibrated
pointcloud imagery where each point are encoded with RGB-value; a depth 
image. The depth images are bound by noise. Hence, the mathematical
model of the map are based on Bayesian probability model:

 \import{\thesisDir/chapter3/equations/}{octomap.tex}

where $P(n)$ are the prior probability distribution, 
$P(n|z_{1:t-1})$ are the previous estimate of the map and $z_t$ 
is the measurement posterior distribution, and $n$, a node in
a segmented space; the map model
encoding of a octree datastructure.
This maps impute the probability value into the segmented space, 
representing occupied and free space. To relax computation, the 
continuous values are conditioned to encode binary values by 
transforming the domain of the random variables $n$ into a logarithmic
scale using logit model. 
The numerator in equation \ref{eq:octomap} are normalizing 
parameter which can be disregarded without having to 
change the mean, mode, and the median of the maps distribution
\parencite{Hornung2013}. 
Hence, together with the relax logit function, equation \ref{eq:octomap}
is reduced to:


\import{\thesisDir/chapter3/equations/}{octomap_relax.tex}

Equation \ref{eq:octomap_relax} are constructed based on the pointcloud
datastructure. The manipulator 
workspace over the octomap model are divided into two values; zero for free
space, and one for occupied space. Since, Realsense also produce RGB 
imageries, the map also valid for the occurance of the color
blue, red, and, green: 

\import{\thesisDir/chapter3/equations/}{octomap_rgb.tex}


With logistic encoding, the datastructure of the map is compact, relaxing
computational expenses and thus eliminate bottleneck 
on the rich streaming from the RGB-D sensor. 
This map model, also known as octomap encodes the posterior of the
SLAM solution in equation \ref{eq:slam_eq}. Given
the stability of binary values in static environment mapping, a 
dynamic objects introduced in the workspace are not represented by
the octomap datastructure.

The SLAM part of this thesis implements the RTAB-Map library
and its framework to encode the map model and also the
spherical harmonics implementation suggested in the earlier section as
the localization engine for the instead the implementation that comes
with the libary. The SLAM model starts with the cost function of the map a posteriori: 

\import{\thesisDir/chapter3/equations/}{slam_model.tex}

\section{Summary}
This chapter introduced the mathematical background that will be adopted and implemented
into the following chapters. The geometry of \rimini~ configuration space, and its 
implicit representation, the control space, are explained. Further, this chapter
introduce the SLAM solution to estimate the task space of the robot. 
%[task]explain we will use Numerical methods to get the Jacobian and test it's determinant
%[task]
%[task]citation for numerical method 
\ifwhole\else
	\printbibliography
	\end{document}
\fi


