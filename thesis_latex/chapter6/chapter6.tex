\providecommand{\setflag}{\newif \ifwhole \wholefalse}
\setflag %create new if variable name ifwhole(whole)  
\ifwhole\else %if whole flag is set to true do nothing else do codes below

  \documentclass[12pt,a4paper,oneside,hidelinks]{book}

    %\input{extra_package.tex}
    %\input{tweak.tex}
    %\input{commando.tex}
    %\input{font}
  \newcommand\thesisDir{/home/asl/version-control/ws_thesis/thesis_latex}
  \usepackage{import}
  %\usepackage{natbib}
  \usepackage[backend=biber,bibencoding=utf8,style=authoryear]{biblatex}
  \addbibresource{\thesisDir/bibtex/library.bib}
	\usepackage{soul,color}
  \usepackage{pdflscape} %for changing page orientation
  \usepackage[shortlabels]{enumitem}
  \usepackage{amssymb}
  \usepackage{graphicx} %for inserting photo
  \usepackage{subcaption}
  \usepackage{amsmath} % for equations
  \DeclareMathOperator*{\argmax}{argmax} % for underword argmax

  \usepackage{mathtools} % for equations
  \usepackage{physics}


  \usepackage{float}  %deactivate this if hyperref does not work with algo
  \usepackage{hyperref}
  \usepackage{tocbibind}  
  \usepackage[linesnumbered,ruled,vlined]{algorithm2e}
  \usepackage{titlesec} %for deeper subsections and changing chpt sec style
  \setcounter{secnumdepth}{4} %setting subsection 4 section deep
  % numbering equations
  \ifwhole
    \numberwithin{equation}{chapter} %equation number based on chapter
  \else
%    \numberwithin{equation}{section} % equation number based on section
  \fi

  \usepackage{xfrac} % for fraction in the for n/d
  \usepackage{esint} % for easy integral symbols

  %for listing algorithms from algorithm2e
  \newcommand{\listofalgorithmes}{\tocfile{\listalgorithmcfname}{loa}}
  %for footnote
  \renewcommand{\thefootnote}{\fnsymbol{footnote}}

  % for formattgin algorithm2e box [iman] I have to put after listofalgorithmes
  \newcommand\mycommfont[1]{\footnotesize\ttfamily\textcolor{blue}{#1}}
  \SetCommentSty{mycommfont}
  \SetKwInput{KwInput}{Input}                % Set the Input
  \SetKwInput{KwOutput}{Output}              % set the Output

  % This command is usefule when importing subdirectory items
  %margin
\usepackage{setspace}
\usepackage[
  a4paper,
  left=3.8cm,
  right=2.5cm,
  top=2.5cm,
  bottom=3cm,
  footskip=1.7cm
]
{geometry}

%IIUM font size
%syntax {\TITLEfontsize IIUM idiotize thesis formatting }
\newcommand\TITLEfontsize{\fontsize{17pt}{20.4pt}\selectfont}
\newcommand\CHAPTERfontsize{\fontsize{14pt}{16pt}\selectfont}
%For line over line for signature
\usepackage{calc}
\newcommand{\sigline}[1]{\makebox[\widthof{#1~}]{.\dotfill}\\#1}
%another over line signature command
%syantax \sign{The signant name} 
%syntax \Date but use it within minipage
%\begin{minipage}[t]{0.4\linewidth}
%    \raggedright
%    \sign{Supervisor}
%    \par
%    Mr.\,L. L. Silva\par
%    Department of Computing and Information Systems, \par
%    Faculty of Applied Sciences, \par
%    University of Moratuwa
%  \end{minipage}%
% \hfill
%  \begin{minipage}[t]{0.4\linewidth}
%    \sign{Signature of the supervisor}
%    \Date
%  \end{minipage}
\newcommand{\sign}[1]{%      
  \begin{tabular}[t]{@{}c@{}}
  \makebox[1.5in]{\dotfill}\\
  \strut#1\strut
  \end{tabular}%
}
\newcommand{\Date}{%
  \begin{tabular}[t]{@{}c@{}}
    \makebox[0.85in]{\dotfill}\\
    \strut Date \strut
  \end{tabular}%
}

%paragraph
\parindent=12mm


%header style
\usepackage{fancyhdr}
%\cfoot{\thepage}
\pagestyle{plain}
%\fancyhead{}
%\fancyhead[C]{\nouppercase{\textit{\leftmark}}} %puts chapter title on even page in lower-case italics
%\fancyhead[CO]{\nouppercase{\textit{\rightmark}}} %puts section title on odd page in lower-case italics
%\renewcommand{\headrulewidth}{0pt} %gets rid of line
\renewcommand{\chaptermark}[1]{\markboth{#1}{}} %gets rid of chapter number
\renewcommand{\sectionmark}[1]{\markright{#1}} %gets rid of section number

%making sure table and figures caption is 1 single spacecaption is 
%make sure that the tables are not in \center environment
%it will add more space between caption and the table/figures
\usepackage{caption}
\captionsetup[table]{skip=12pt}

%adding frames to a page for copygright
\usepackage{mdframed}
\usepackage{verbatim}

%using font close to Times New Roman
\usepackage{mathptmx}
% since mathptmx changes \mathcal symbols 
% use this patch to retained the default one
\DeclareMathAlphabet{\mathcal}{OMS}{cmsy}{m}{n}


\usepackage{titletoc}

%provide control over the appearance of table of contents
%figures and others
\usepackage[titles]{tocloft}
%adding dotted or dot leaders to chapter in table of content (toc)
%\renewcommand{\cftchapfont}{\bfseries}
%\renewcommand{\cftchappagefont}{\bfseries}
\renewcommand{\cftchapleader}{\cftdotfill{\cftdotsep}} % for chapters
% prepend CHAPTER on chapter number in toc
\renewcommand{\cftchappresnum}{CHAPTER }
% add ':' separator between chapter number and chapter title
\renewcommand{\cftchapaftersnum}{:}
\setlength{\cftchapnumwidth}{6.5em}
% making section normal font instead of bold
%adding "Table" keyword to the list
%provide abit of space so that the 'Table X.XX'
%keyword does not overlap with the table caption
\renewcommand{\cfttabpresnum}{Table\ }
\setlength{\cfttabnumwidth}{5.5em}

%adding the "Figure" keyword to the list
%provide abit of space so that the 'Figure X.XX'
%keyword does not overlap with the figure caption
\renewcommand{\cftfigpresnum}{Figure\ }
\setlength{\cftfignumwidth}{5.5em}

% Annotation is being displayed in bibliography/references
% This will suppress it
\DeclareSourcemap{
  \maps[datatype=bibtex]{
    \map{
      \step[fieldset=annote,     null]
      \step[fieldset=annotation, null]
    }
  }
}


  
  %This is for group notation
   \newcommand\thisPaperDir{/home/iman/version-control/ws_thesis/writing_papers/resampling_planning_in_dynamic_environment}
 
\begin{document}

\fi


\chapter{Conclusion And Recommendation}
\label{chap:conclusion_recommendation}

\section{Conclusion}
    In this research, a prototype of an industrial robot is 
    developed to investigate the planning and motion control 
    of the manipulator for compliant usage in an industrial 
    setup.

    The robot is as 6R manipulator, with six-degree-of-freedom. Each joints is actuated 
    with Dynamixel servo and are back-drivable. The end-effector is equipped with 
    an RGB-D sensor. The robot is named \rimini~. 
    
    Since the Dynamixel motors are not supplemented with a mathematical model, the joints
    are controlled based on time-parameterized controller where, the set up of each of
    the motor's velocity profile depends on the angular velocity limits and the angular
    acceleration limit. The time-parameterized controller was successfully tuned with informed velocity-acceleration limit
    parameterization. All of the \rimini~   controllers
    parameters and system configuration, including it's driver, are package as 
    a stack of ROS packages.

A benchmark was done to ascertain the best sampling-based planner for the \rimini~'s 
capability to avoid moving obstacles. The simulation for the benchmark considers a
static object, placed in the manipulator's workspace. 
RRT was selected given it's
rapid processing time, 0.031 s, at solving a planning objective. 
Two simulations were done; the first involved  
introducing a static object with 
unpredictable shape and placement 
into the manipulators line of motion, and the second 
simulation setup involve a moving object with the shape of a cylinder that was placed and moved in the robot's workspace.

Both the simulations were validated with \rimini~ hardware. 
The unpredictable static object invoked reactive motion 
correction, where no collision were reported.  
However, when moving obstacles were introduced, the replanning
fail to provide collision-free solution. The robot capability to avoid the moving obstacle, although less successful,
has not been consistent. The avoidance fails if
the RRT is invoke when the obstacle nearly approaching the $C_{cycle}$.

Although the performance of the RRT on a dynamic 
obstacle imposed under the cyclical space prescribed in algorithm \ref{algo:cycle_space}, 
is not satisfactory, the result shows that the robot
is capable at reacting to an obstacles when the obstacle 
is moving.
    
    This thesis's SLAM implementation, by repurposing
RTAB-Map as the SLAM framework and PHASER repurposing as the state estimation pipeline of the RTAB-Map, 
shows an intermittent and sparse estimation of the $C_{ee}$ which fails to 
continuously estimate the joint-configuration of the manipulator. 


\section{Recommendation for Future Works}
This thesis recommends a future work on improved state estimations of the 
RTAB-Map where the singularities reading during state estimation can be pass to
a splining process. The splining would consider the last reading of the RTAB-Map 
estimation pipeline at, $t_{last}$, and the output from the 
\textit{equation \ref{eq:splining_traj_selection}},

\import{\thesisDir/chapter6/equations/}{splining_traj_selection.tex}

where $\hat{C}_{last}$ is the last state estimation of the RTAB-Map before data silence.
This approach has the potential of providing pseudo-continuous streaming of feedback for
joint controllers; follow-joint-trajectory controller in ROS
being the immediate candidate for an encoder-less context. The implication
is an encoder-less system that requires no encoders as feedback and, hence, reduces
the cost, space of the robot design, and the market price for commercialization of
affordable automation system to SME's. 

\ifwhole\else
	\printbibliography
	\end{document}
\fi


